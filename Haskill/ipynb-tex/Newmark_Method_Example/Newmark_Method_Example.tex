\documentclass[11pt]{article}

    \usepackage[breakable]{tcolorbox}
    \usepackage{parskip} % Stop auto-indenting (to mimic markdown behaviour)
    
    % 添加中文支持
    \usepackage{xeCJK}
    \setCJKmainfont{STSong} % 使用华文宋体,您也可以换成其他中文字体
    
    % 确保使用 XeLaTeX 编译器
    \usepackage{ifxetex}
    \RequireXeTeX

    

    % Basic figure setup, for now with no caption control since it's done
    % automatically by Pandoc (which extracts ![](path) syntax from Markdown).
    \usepackage{graphicx}
    % Keep aspect ratio if custom image width or height is specified
    \setkeys{Gin}{keepaspectratio}
    % Maintain compatibility with old templates. Remove in nbconvert 6.0
    \let\Oldincludegraphics\includegraphics
    % Ensure that by default, figures have no caption (until we provide a
    % proper Figure object with a Caption API and a way to capture that
    % in the conversion process - todo).
    \usepackage{caption}
    \DeclareCaptionFormat{nocaption}{}
    \captionsetup{format=nocaption,aboveskip=0pt,belowskip=0pt}

    \usepackage{float}
    \floatplacement{figure}{H} % forces figures to be placed at the correct location
    \usepackage{xcolor} % Allow colors to be defined
    \usepackage{enumerate} % Needed for markdown enumerations to work
    \usepackage{geometry} % Used to adjust the document margins
    \usepackage{amsmath} % Equations
    \usepackage{amssymb} % Equations
    \usepackage{textcomp} % defines textquotesingle
    % Hack from http://tex.stackexchange.com/a/47451/13684:
    \AtBeginDocument{%
        \def\PYZsq{\textquotesingle}% Upright quotes in Pygmentized code
    }
    \usepackage{upquote} % Upright quotes for verbatim code
    \usepackage{eurosym} % defines \euro

    \usepackage{iftex}
    \ifPDFTeX
        \usepackage[T1]{fontenc}
        \IfFileExists{alphabeta.sty}{
              \usepackage{alphabeta}
          }{
              \usepackage[mathletters]{ucs}
              \usepackage[utf8x]{inputenc}
          }
    \else
        \usepackage{fontspec}
        \usepackage{unicode-math}
    \fi

    \usepackage{fancyvrb} % verbatim replacement that allows latex
    \usepackage{grffile} % extends the file name processing of package graphics
                         % to support a larger range
    \makeatletter % fix for old versions of grffile with XeLaTeX
    \@ifpackagelater{grffile}{2019/11/01}
    {
      % Do nothing on new versions
    }
    {
      \def\Gread@@xetex#1{%
        \IfFileExists{"\Gin@base".bb}%
        {\Gread@eps{\Gin@base.bb}}%
        {\Gread@@xetex@aux#1}%
      }
    }
    \makeatother
    \usepackage[Export]{adjustbox} % Used to constrain images to a maximum size
    \adjustboxset{max size={0.9\linewidth}{0.9\paperheight}}

    % The hyperref package gives us a pdf with properly built
    % internal navigation ('pdf bookmarks' for the table of contents,
    % internal cross-reference links, web links for URLs, etc.)
    \usepackage{hyperref}
    % The default LaTeX title has an obnoxious amount of whitespace. By default,
    % titling removes some of it. It also provides customization options.
    \usepackage{titling}
    \usepackage{longtable} % longtable support required by pandoc >1.10
    \usepackage{booktabs}  % table support for pandoc > 1.12.2
    \usepackage{array}     % table support for pandoc >= 2.11.3
    \usepackage{calc}      % table minipage width calculation for pandoc >= 2.11.1
    \usepackage[inline]{enumitem} % IRkernel/repr support (it uses the enumerate* environment)
    \usepackage[normalem]{ulem} % ulem is needed to support strikethroughs (\sout)
                                % normalem makes italics be italics, not underlines
    \usepackage{soul}      % strikethrough (\st) support for pandoc >= 3.0.0
    \usepackage{mathrsfs}
    

    
    % Colors for the hyperref package
    \definecolor{urlcolor}{rgb}{0,.145,.698}
    \definecolor{linkcolor}{rgb}{.71,0.21,0.01}
    \definecolor{citecolor}{rgb}{.12,.54,.11}

    % ANSI colors
    \definecolor{ansi-black}{HTML}{3E424D}
    \definecolor{ansi-black-intense}{HTML}{282C36}
    \definecolor{ansi-red}{HTML}{E75C58}
    \definecolor{ansi-red-intense}{HTML}{B22B31}
    \definecolor{ansi-green}{HTML}{00A250}
    \definecolor{ansi-green-intense}{HTML}{007427}
    \definecolor{ansi-yellow}{HTML}{DDB62B}
    \definecolor{ansi-yellow-intense}{HTML}{B27D12}
    \definecolor{ansi-blue}{HTML}{208FFB}
    \definecolor{ansi-blue-intense}{HTML}{0065CA}
    \definecolor{ansi-magenta}{HTML}{D160C4}
    \definecolor{ansi-magenta-intense}{HTML}{A03196}
    \definecolor{ansi-cyan}{HTML}{60C6C8}
    \definecolor{ansi-cyan-intense}{HTML}{258F8F}
    \definecolor{ansi-white}{HTML}{C5C1B4}
    \definecolor{ansi-white-intense}{HTML}{A1A6B2}
    \definecolor{ansi-default-inverse-fg}{HTML}{FFFFFF}
    \definecolor{ansi-default-inverse-bg}{HTML}{000000}

    % common color for the border for error outputs.
    \definecolor{outerrorbackground}{HTML}{FFDFDF}

    % commands and environments needed by pandoc snippets
    % extracted from the output of `pandoc -s`
    \providecommand{\tightlist}{%
      \setlength{\itemsep}{0pt}\setlength{\parskip}{0pt}}
    \DefineVerbatimEnvironment{Highlighting}{Verbatim}{commandchars=\\\{\}}
    % Add ',fontsize=\small' for more characters per line
    \newenvironment{Shaded}{}{}
    \newcommand{\KeywordTok}[1]{\textcolor[rgb]{0.00,0.44,0.13}{\textbf{{#1}}}}
    \newcommand{\DataTypeTok}[1]{\textcolor[rgb]{0.56,0.13,0.00}{{#1}}}
    \newcommand{\DecValTok}[1]{\textcolor[rgb]{0.25,0.63,0.44}{{#1}}}
    \newcommand{\BaseNTok}[1]{\textcolor[rgb]{0.25,0.63,0.44}{{#1}}}
    \newcommand{\FloatTok}[1]{\textcolor[rgb]{0.25,0.63,0.44}{{#1}}}
    \newcommand{\CharTok}[1]{\textcolor[rgb]{0.25,0.44,0.63}{{#1}}}
    \newcommand{\StringTok}[1]{\textcolor[rgb]{0.25,0.44,0.63}{{#1}}}
    \newcommand{\CommentTok}[1]{\textcolor[rgb]{0.38,0.63,0.69}{\textit{{#1}}}}
    \newcommand{\OtherTok}[1]{\textcolor[rgb]{0.00,0.44,0.13}{{#1}}}
    \newcommand{\AlertTok}[1]{\textcolor[rgb]{1.00,0.00,0.00}{\textbf{{#1}}}}
    \newcommand{\FunctionTok}[1]{\textcolor[rgb]{0.02,0.16,0.49}{{#1}}}
    \newcommand{\RegionMarkerTok}[1]{{#1}}
    \newcommand{\ErrorTok}[1]{\textcolor[rgb]{1.00,0.00,0.00}{\textbf{{#1}}}}
    \newcommand{\NormalTok}[1]{{#1}}

    % Additional commands for more recent versions of Pandoc
    \newcommand{\ConstantTok}[1]{\textcolor[rgb]{0.53,0.00,0.00}{{#1}}}
    \newcommand{\SpecialCharTok}[1]{\textcolor[rgb]{0.25,0.44,0.63}{{#1}}}
    \newcommand{\VerbatimStringTok}[1]{\textcolor[rgb]{0.25,0.44,0.63}{{#1}}}
    \newcommand{\SpecialStringTok}[1]{\textcolor[rgb]{0.73,0.40,0.53}{{#1}}}
    \newcommand{\ImportTok}[1]{{#1}}
    \newcommand{\DocumentationTok}[1]{\textcolor[rgb]{0.73,0.13,0.13}{\textit{{#1}}}}
    \newcommand{\AnnotationTok}[1]{\textcolor[rgb]{0.38,0.63,0.69}{\textbf{\textit{{#1}}}}}
    \newcommand{\CommentVarTok}[1]{\textcolor[rgb]{0.38,0.63,0.69}{\textbf{\textit{{#1}}}}}
    \newcommand{\VariableTok}[1]{\textcolor[rgb]{0.10,0.09,0.49}{{#1}}}
    \newcommand{\ControlFlowTok}[1]{\textcolor[rgb]{0.00,0.44,0.13}{\textbf{{#1}}}}
    \newcommand{\OperatorTok}[1]{\textcolor[rgb]{0.40,0.40,0.40}{{#1}}}
    \newcommand{\BuiltInTok}[1]{{#1}}
    \newcommand{\ExtensionTok}[1]{{#1}}
    \newcommand{\PreprocessorTok}[1]{\textcolor[rgb]{0.74,0.48,0.00}{{#1}}}
    \newcommand{\AttributeTok}[1]{\textcolor[rgb]{0.49,0.56,0.16}{{#1}}}
    \newcommand{\InformationTok}[1]{\textcolor[rgb]{0.38,0.63,0.69}{\textbf{\textit{{#1}}}}}
    \newcommand{\WarningTok}[1]{\textcolor[rgb]{0.38,0.63,0.69}{\textbf{\textit{{#1}}}}}


    % Define a nice break command that doesn't care if a line doesn't already
    % exist.
    \def\br{\hspace*{\fill} \\* }
    % Math Jax compatibility definitions
    \def\gt{>}
    \def\lt{<}
    \let\Oldtex\TeX
    \let\Oldlatex\LaTeX
    \renewcommand{\TeX}{\textrm{\Oldtex}}
    \renewcommand{\LaTeX}{\textrm{\Oldlatex}}
    % Document parameters
    % Document title
    \title{Newmark\_Method\_Example}
    
    
    
    
    
    
    
% Pygments definitions
\makeatletter
\def\PY@reset{\let\PY@it=\relax \let\PY@bf=\relax%
    \let\PY@ul=\relax \let\PY@tc=\relax%
    \let\PY@bc=\relax \let\PY@ff=\relax}
\def\PY@tok#1{\csname PY@tok@#1\endcsname}
\def\PY@toks#1+{\ifx\relax#1\empty\else%
    \PY@tok{#1}\expandafter\PY@toks\fi}
\def\PY@do#1{\PY@bc{\PY@tc{\PY@ul{%
    \PY@it{\PY@bf{\PY@ff{#1}}}}}}}
\def\PY#1#2{\PY@reset\PY@toks#1+\relax+\PY@do{#2}}

\@namedef{PY@tok@w}{\def\PY@tc##1{\textcolor[rgb]{0.73,0.73,0.73}{##1}}}
\@namedef{PY@tok@c}{\let\PY@it=\textit\def\PY@tc##1{\textcolor[rgb]{0.24,0.48,0.48}{##1}}}
\@namedef{PY@tok@cp}{\def\PY@tc##1{\textcolor[rgb]{0.61,0.40,0.00}{##1}}}
\@namedef{PY@tok@k}{\let\PY@bf=\textbf\def\PY@tc##1{\textcolor[rgb]{0.00,0.50,0.00}{##1}}}
\@namedef{PY@tok@kp}{\def\PY@tc##1{\textcolor[rgb]{0.00,0.50,0.00}{##1}}}
\@namedef{PY@tok@kt}{\def\PY@tc##1{\textcolor[rgb]{0.69,0.00,0.25}{##1}}}
\@namedef{PY@tok@o}{\def\PY@tc##1{\textcolor[rgb]{0.40,0.40,0.40}{##1}}}
\@namedef{PY@tok@ow}{\let\PY@bf=\textbf\def\PY@tc##1{\textcolor[rgb]{0.67,0.13,1.00}{##1}}}
\@namedef{PY@tok@nb}{\def\PY@tc##1{\textcolor[rgb]{0.00,0.50,0.00}{##1}}}
\@namedef{PY@tok@nf}{\def\PY@tc##1{\textcolor[rgb]{0.00,0.00,1.00}{##1}}}
\@namedef{PY@tok@nc}{\let\PY@bf=\textbf\def\PY@tc##1{\textcolor[rgb]{0.00,0.00,1.00}{##1}}}
\@namedef{PY@tok@nn}{\let\PY@bf=\textbf\def\PY@tc##1{\textcolor[rgb]{0.00,0.00,1.00}{##1}}}
\@namedef{PY@tok@ne}{\let\PY@bf=\textbf\def\PY@tc##1{\textcolor[rgb]{0.80,0.25,0.22}{##1}}}
\@namedef{PY@tok@nv}{\def\PY@tc##1{\textcolor[rgb]{0.10,0.09,0.49}{##1}}}
\@namedef{PY@tok@no}{\def\PY@tc##1{\textcolor[rgb]{0.53,0.00,0.00}{##1}}}
\@namedef{PY@tok@nl}{\def\PY@tc##1{\textcolor[rgb]{0.46,0.46,0.00}{##1}}}
\@namedef{PY@tok@ni}{\let\PY@bf=\textbf\def\PY@tc##1{\textcolor[rgb]{0.44,0.44,0.44}{##1}}}
\@namedef{PY@tok@na}{\def\PY@tc##1{\textcolor[rgb]{0.41,0.47,0.13}{##1}}}
\@namedef{PY@tok@nt}{\let\PY@bf=\textbf\def\PY@tc##1{\textcolor[rgb]{0.00,0.50,0.00}{##1}}}
\@namedef{PY@tok@nd}{\def\PY@tc##1{\textcolor[rgb]{0.67,0.13,1.00}{##1}}}
\@namedef{PY@tok@s}{\def\PY@tc##1{\textcolor[rgb]{0.73,0.13,0.13}{##1}}}
\@namedef{PY@tok@sd}{\let\PY@it=\textit\def\PY@tc##1{\textcolor[rgb]{0.73,0.13,0.13}{##1}}}
\@namedef{PY@tok@si}{\let\PY@bf=\textbf\def\PY@tc##1{\textcolor[rgb]{0.64,0.35,0.47}{##1}}}
\@namedef{PY@tok@se}{\let\PY@bf=\textbf\def\PY@tc##1{\textcolor[rgb]{0.67,0.36,0.12}{##1}}}
\@namedef{PY@tok@sr}{\def\PY@tc##1{\textcolor[rgb]{0.64,0.35,0.47}{##1}}}
\@namedef{PY@tok@ss}{\def\PY@tc##1{\textcolor[rgb]{0.10,0.09,0.49}{##1}}}
\@namedef{PY@tok@sx}{\def\PY@tc##1{\textcolor[rgb]{0.00,0.50,0.00}{##1}}}
\@namedef{PY@tok@m}{\def\PY@tc##1{\textcolor[rgb]{0.40,0.40,0.40}{##1}}}
\@namedef{PY@tok@gh}{\let\PY@bf=\textbf\def\PY@tc##1{\textcolor[rgb]{0.00,0.00,0.50}{##1}}}
\@namedef{PY@tok@gu}{\let\PY@bf=\textbf\def\PY@tc##1{\textcolor[rgb]{0.50,0.00,0.50}{##1}}}
\@namedef{PY@tok@gd}{\def\PY@tc##1{\textcolor[rgb]{0.63,0.00,0.00}{##1}}}
\@namedef{PY@tok@gi}{\def\PY@tc##1{\textcolor[rgb]{0.00,0.52,0.00}{##1}}}
\@namedef{PY@tok@gr}{\def\PY@tc##1{\textcolor[rgb]{0.89,0.00,0.00}{##1}}}
\@namedef{PY@tok@ge}{\let\PY@it=\textit}
\@namedef{PY@tok@gs}{\let\PY@bf=\textbf}
\@namedef{PY@tok@gp}{\let\PY@bf=\textbf\def\PY@tc##1{\textcolor[rgb]{0.00,0.00,0.50}{##1}}}
\@namedef{PY@tok@go}{\def\PY@tc##1{\textcolor[rgb]{0.44,0.44,0.44}{##1}}}
\@namedef{PY@tok@gt}{\def\PY@tc##1{\textcolor[rgb]{0.00,0.27,0.87}{##1}}}
\@namedef{PY@tok@err}{\def\PY@bc##1{{\setlength{\fboxsep}{\string -\fboxrule}\fcolorbox[rgb]{1.00,0.00,0.00}{1,1,1}{\strut ##1}}}}
\@namedef{PY@tok@kc}{\let\PY@bf=\textbf\def\PY@tc##1{\textcolor[rgb]{0.00,0.50,0.00}{##1}}}
\@namedef{PY@tok@kd}{\let\PY@bf=\textbf\def\PY@tc##1{\textcolor[rgb]{0.00,0.50,0.00}{##1}}}
\@namedef{PY@tok@kn}{\let\PY@bf=\textbf\def\PY@tc##1{\textcolor[rgb]{0.00,0.50,0.00}{##1}}}
\@namedef{PY@tok@kr}{\let\PY@bf=\textbf\def\PY@tc##1{\textcolor[rgb]{0.00,0.50,0.00}{##1}}}
\@namedef{PY@tok@bp}{\def\PY@tc##1{\textcolor[rgb]{0.00,0.50,0.00}{##1}}}
\@namedef{PY@tok@fm}{\def\PY@tc##1{\textcolor[rgb]{0.00,0.00,1.00}{##1}}}
\@namedef{PY@tok@vc}{\def\PY@tc##1{\textcolor[rgb]{0.10,0.09,0.49}{##1}}}
\@namedef{PY@tok@vg}{\def\PY@tc##1{\textcolor[rgb]{0.10,0.09,0.49}{##1}}}
\@namedef{PY@tok@vi}{\def\PY@tc##1{\textcolor[rgb]{0.10,0.09,0.49}{##1}}}
\@namedef{PY@tok@vm}{\def\PY@tc##1{\textcolor[rgb]{0.10,0.09,0.49}{##1}}}
\@namedef{PY@tok@sa}{\def\PY@tc##1{\textcolor[rgb]{0.73,0.13,0.13}{##1}}}
\@namedef{PY@tok@sb}{\def\PY@tc##1{\textcolor[rgb]{0.73,0.13,0.13}{##1}}}
\@namedef{PY@tok@sc}{\def\PY@tc##1{\textcolor[rgb]{0.73,0.13,0.13}{##1}}}
\@namedef{PY@tok@dl}{\def\PY@tc##1{\textcolor[rgb]{0.73,0.13,0.13}{##1}}}
\@namedef{PY@tok@s2}{\def\PY@tc##1{\textcolor[rgb]{0.73,0.13,0.13}{##1}}}
\@namedef{PY@tok@sh}{\def\PY@tc##1{\textcolor[rgb]{0.73,0.13,0.13}{##1}}}
\@namedef{PY@tok@s1}{\def\PY@tc##1{\textcolor[rgb]{0.73,0.13,0.13}{##1}}}
\@namedef{PY@tok@mb}{\def\PY@tc##1{\textcolor[rgb]{0.40,0.40,0.40}{##1}}}
\@namedef{PY@tok@mf}{\def\PY@tc##1{\textcolor[rgb]{0.40,0.40,0.40}{##1}}}
\@namedef{PY@tok@mh}{\def\PY@tc##1{\textcolor[rgb]{0.40,0.40,0.40}{##1}}}
\@namedef{PY@tok@mi}{\def\PY@tc##1{\textcolor[rgb]{0.40,0.40,0.40}{##1}}}
\@namedef{PY@tok@il}{\def\PY@tc##1{\textcolor[rgb]{0.40,0.40,0.40}{##1}}}
\@namedef{PY@tok@mo}{\def\PY@tc##1{\textcolor[rgb]{0.40,0.40,0.40}{##1}}}
\@namedef{PY@tok@ch}{\let\PY@it=\textit\def\PY@tc##1{\textcolor[rgb]{0.24,0.48,0.48}{##1}}}
\@namedef{PY@tok@cm}{\let\PY@it=\textit\def\PY@tc##1{\textcolor[rgb]{0.24,0.48,0.48}{##1}}}
\@namedef{PY@tok@cpf}{\let\PY@it=\textit\def\PY@tc##1{\textcolor[rgb]{0.24,0.48,0.48}{##1}}}
\@namedef{PY@tok@c1}{\let\PY@it=\textit\def\PY@tc##1{\textcolor[rgb]{0.24,0.48,0.48}{##1}}}
\@namedef{PY@tok@cs}{\let\PY@it=\textit\def\PY@tc##1{\textcolor[rgb]{0.24,0.48,0.48}{##1}}}

\def\PYZbs{\char`\\}
\def\PYZus{\char`\_}
\def\PYZob{\char`\{}
\def\PYZcb{\char`\}}
\def\PYZca{\char`\^}
\def\PYZam{\char`\&}
\def\PYZlt{\char`\<}
\def\PYZgt{\char`\>}
\def\PYZsh{\char`\#}
\def\PYZpc{\char`\%}
\def\PYZdl{\char`\$}
\def\PYZhy{\char`\-}
\def\PYZsq{\char`\'}
\def\PYZdq{\char`\"}
\def\PYZti{\char`\~}
% for compatibility with earlier versions
\def\PYZat{@}
\def\PYZlb{[}
\def\PYZrb{]}
\makeatother


    % For linebreaks inside Verbatim environment from package fancyvrb.
    \makeatletter
        \newbox\Wrappedcontinuationbox
        \newbox\Wrappedvisiblespacebox
        \newcommand*\Wrappedvisiblespace {\textcolor{red}{\textvisiblespace}}
        \newcommand*\Wrappedcontinuationsymbol {\textcolor{red}{\llap{\tiny$\m@th\hookrightarrow$}}}
        \newcommand*\Wrappedcontinuationindent {3ex }
        \newcommand*\Wrappedafterbreak {\kern\Wrappedcontinuationindent\copy\Wrappedcontinuationbox}
        % Take advantage of the already applied Pygments mark-up to insert
        % potential linebreaks for TeX processing.
        %        {, <, #, %, $, ' and ": go to next line.
        %        _, }, ^, &, >, - and ~: stay at end of broken line.
        % Use of \textquotesingle for straight quote.
        \newcommand*\Wrappedbreaksatspecials {%
            \def\PYGZus{\discretionary{\char`\_}{\Wrappedafterbreak}{\char`\_}}%
            \def\PYGZob{\discretionary{}{\Wrappedafterbreak\char`\{}{\char`\{}}%
            \def\PYGZcb{\discretionary{\char`\}}{\Wrappedafterbreak}{\char`\}}}%
            \def\PYGZca{\discretionary{\char`\^}{\Wrappedafterbreak}{\char`\^}}%
            \def\PYGZam{\discretionary{\char`\&}{\Wrappedafterbreak}{\char`\&}}%
            \def\PYGZlt{\discretionary{}{\Wrappedafterbreak\char`\<}{\char`\<}}%
            \def\PYGZgt{\discretionary{\char`\>}{\Wrappedafterbreak}{\char`\>}}%
            \def\PYGZsh{\discretionary{}{\Wrappedafterbreak\char`\#}{\char`\#}}%
            \def\PYGZpc{\discretionary{}{\Wrappedafterbreak\char`\%}{\char`\%}}%
            \def\PYGZdl{\discretionary{}{\Wrappedafterbreak\char`\$}{\char`\$}}%
            \def\PYGZhy{\discretionary{\char`\-}{\Wrappedafterbreak}{\char`\-}}%
            \def\PYGZsq{\discretionary{}{\Wrappedafterbreak\textquotesingle}{\textquotesingle}}%
            \def\PYGZdq{\discretionary{}{\Wrappedafterbreak\char`\"}{\char`\"}}%
            \def\PYGZti{\discretionary{\char`\~}{\Wrappedafterbreak}{\char`\~}}%
        }
        % Some characters . , ; ? ! / are not pygmentized.
        % This macro makes them "active" and they will insert potential linebreaks
        \newcommand*\Wrappedbreaksatpunct {%
            \lccode`\~`\.\lowercase{\def~}{\discretionary{\hbox{\char`\.}}{\Wrappedafterbreak}{\hbox{\char`\.}}}%
            \lccode`\~`\,\lowercase{\def~}{\discretionary{\hbox{\char`\,}}{\Wrappedafterbreak}{\hbox{\char`\,}}}%
            \lccode`\~`\;\lowercase{\def~}{\discretionary{\hbox{\char`\;}}{\Wrappedafterbreak}{\hbox{\char`\;}}}%
            \lccode`\~`\:\lowercase{\def~}{\discretionary{\hbox{\char`\:}}{\Wrappedafterbreak}{\hbox{\char`\:}}}%
            \lccode`\~`\?\lowercase{\def~}{\discretionary{\hbox{\char`\?}}{\Wrappedafterbreak}{\hbox{\char`\?}}}%
            \lccode`\~`\!\lowercase{\def~}{\discretionary{\hbox{\char`\!}}{\Wrappedafterbreak}{\hbox{\char`\!}}}%
            \lccode`\~`\/\lowercase{\def~}{\discretionary{\hbox{\char`\/}}{\Wrappedafterbreak}{\hbox{\char`\/}}}%
            \catcode`\.\active
            \catcode`\,\active
            \catcode`\;\active
            \catcode`\:\active
            \catcode`\?\active
            \catcode`\!\active
            \catcode`\/\active
            \lccode`\~`\~
        }
    \makeatother

    \let\OriginalVerbatim=\Verbatim
    \makeatletter
    \renewcommand{\Verbatim}[1][1]{%
        %\parskip\z@skip
        \sbox\Wrappedcontinuationbox {\Wrappedcontinuationsymbol}%
        \sbox\Wrappedvisiblespacebox {\FV@SetupFont\Wrappedvisiblespace}%
        \def\FancyVerbFormatLine ##1{\hsize\linewidth
            \vtop{\raggedright\hyphenpenalty\z@\exhyphenpenalty\z@
                \doublehyphendemerits\z@\finalhyphendemerits\z@
                \strut ##1\strut}%
        }%
        % If the linebreak is at a space, the latter will be displayed as visible
        % space at end of first line, and a continuation symbol starts next line.
        % Stretch/shrink are however usually zero for typewriter font.
        \def\FV@Space {%
            \nobreak\hskip\z@ plus\fontdimen3\font minus\fontdimen4\font
            \discretionary{\copy\Wrappedvisiblespacebox}{\Wrappedafterbreak}
            {\kern\fontdimen2\font}%
        }%

        % Allow breaks at special characters using \PYG... macros.
        \Wrappedbreaksatspecials
        % Breaks at punctuation characters . , ; ? ! and / need catcode=\active
        \OriginalVerbatim[#1,codes*=\Wrappedbreaksatpunct]%
    }
    \makeatother

    % Exact colors from NB
    \definecolor{incolor}{HTML}{303F9F}
    \definecolor{outcolor}{HTML}{D84315}
    \definecolor{cellborder}{HTML}{CFCFCF}
    \definecolor{cellbackground}{HTML}{F7F7F7}

    % prompt
    \makeatletter
    \newcommand{\boxspacing}{\kern\kvtcb@left@rule\kern\kvtcb@boxsep}
    \makeatother
    \newcommand{\prompt}[4]{
        {\ttfamily\llap{{\color{#2}[#3]:\hspace{3pt}#4}}\vspace{-\baselineskip}}
    }
    

    
    % Prevent overflowing lines due to hard-to-break entities
    \sloppy
    % Setup hyperref package
    \hypersetup{
      breaklinks=true,  % so long urls are correctly broken across lines
      colorlinks=true,
      urlcolor=urlcolor,
      linkcolor=linkcolor,
      citecolor=citecolor,
      }
    % Slightly bigger margins than the latex defaults
    
    \geometry{verbose,tmargin=1in,bmargin=1in,lmargin=1in,rmargin=1in}
    
    

\begin{document}
    
    \maketitle
    
    

    
    \section{Newmark数值积分法详解与算例}\label{newmarkux6570ux503cux79efux5206ux6cd5ux8be6ux89e3ux4e0eux7b97ux4f8b}

Newmark方法是结构动力学中常用的时间步积分方法,用于求解结构的动力响应。该方法由N.M.
Newmark于1959年提出,是一类隐式积分方法,广泛应用于地震工程和结构动力学分析中。

    \subsection{1.
Newmark方法的理论基础}\label{newmarkux65b9ux6cd5ux7684ux7406ux8bbaux57faux7840}

Newmark方法基于以下位移和速度的迭代公式:

\[u_{n+1} = u_n + \Delta t \dot{u}_n + \frac{\Delta t^2}{2}[(1-2\beta)\ddot{u}_n + 2\beta\ddot{u}_{n+1}]\]

\[\dot{u}_{n+1} = \dot{u}_n + \Delta t[(1-\gamma)\ddot{u}_n + \gamma\ddot{u}_{n+1}]\]

其中: - \(u\), \(\dot{u}\), \(\ddot{u}\) 分别表示位移、速度和加速度 -
\(\Delta t\) 是时间步长 - \(\beta\) 和 \(\gamma\)
是控制积分精度和稳定性的参数 - 当 \(\gamma = 1/2, \beta = 1/4\)
时,为平均加速度法(无条件稳定) - 当 \(\gamma = 1/2, \beta = 1/6\)
时,为线性加速度法(条件稳定)

    \subsection{2.
单自由度系统的Newmark方法求解}\label{ux5355ux81eaux7531ux5ea6ux7cfbux7edfux7684newmarkux65b9ux6cd5ux6c42ux89e3}

单自由度系统的运动方程为:

\[m\ddot{u} + c\dot{u} + ku = p(t)\]

其中: - \(m\) 是质量 - \(c\) 是阻尼系数 - \(k\) 是刚度 - \(p(t)\)
是外部荷载

下面我们将导入必要的库并设置一个算例来演示Newmark方法的应用。

    \begin{tcolorbox}[breakable, size=fbox, boxrule=1pt, pad at break*=1mm,colback=cellbackground, colframe=cellborder]
\prompt{In}{incolor}{1}{\boxspacing}
\begin{Verbatim}[commandchars=\\\{\}]
\PY{k+kn}{import} \PY{n+nn}{numpy} \PY{k}{as} \PY{n+nn}{np}
\PY{k+kn}{import} \PY{n+nn}{matplotlib}\PY{n+nn}{.}\PY{n+nn}{pyplot} \PY{k}{as} \PY{n+nn}{plt}
\PY{k+kn}{from} \PY{n+nn}{scipy}\PY{n+nn}{.}\PY{n+nn}{integrate} \PY{k+kn}{import} \PY{n}{solve\PYZus{}ivp}

\PY{c+c1}{\PYZsh{} 设置绘图样式}
\PY{n}{plt}\PY{o}{.}\PY{n}{style}\PY{o}{.}\PY{n}{use}\PY{p}{(}\PY{l+s+s1}{\PYZsq{}}\PY{l+s+s1}{seaborn\PYZhy{}v0\PYZus{}8\PYZhy{}whitegrid}\PY{l+s+s1}{\PYZsq{}}\PY{p}{)}
\PY{n}{plt}\PY{o}{.}\PY{n}{rcParams}\PY{p}{[}\PY{l+s+s1}{\PYZsq{}}\PY{l+s+s1}{figure.figsize}\PY{l+s+s1}{\PYZsq{}}\PY{p}{]} \PY{o}{=} \PY{p}{(}\PY{l+m+mi}{12}\PY{p}{,} \PY{l+m+mi}{8}\PY{p}{)}
\PY{n}{plt}\PY{o}{.}\PY{n}{rcParams}\PY{p}{[}\PY{l+s+s1}{\PYZsq{}}\PY{l+s+s1}{font.size}\PY{l+s+s1}{\PYZsq{}}\PY{p}{]} \PY{o}{=} \PY{l+m+mi}{12}
\end{Verbatim}
\end{tcolorbox}

    \subsection{3. 算例设置 -
单自由度系统}\label{ux7b97ux4f8bux8bbeux7f6e---ux5355ux81eaux7531ux5ea6ux7cfbux7edf}

我们考虑一个受简谐外力作用的单自由度弹簧-质量-阻尼系统。该系统的参数如下:

    \begin{tcolorbox}[breakable, size=fbox, boxrule=1pt, pad at break*=1mm,colback=cellbackground, colframe=cellborder]
\prompt{In}{incolor}{2}{\boxspacing}
\begin{Verbatim}[commandchars=\\\{\}]
\PY{c+c1}{\PYZsh{} 系统参数}
\PY{n}{m} \PY{o}{=} \PY{l+m+mf}{1.0}       \PY{c+c1}{\PYZsh{} 质量 (kg)}
\PY{n}{k} \PY{o}{=} \PY{l+m+mf}{100.0}     \PY{c+c1}{\PYZsh{} 刚度 (N/m)}
\PY{n}{xi} \PY{o}{=} \PY{l+m+mf}{0.05}     \PY{c+c1}{\PYZsh{} 阻尼比}
\PY{n}{omega\PYZus{}n} \PY{o}{=} \PY{n}{np}\PY{o}{.}\PY{n}{sqrt}\PY{p}{(}\PY{n}{k}\PY{o}{/}\PY{n}{m}\PY{p}{)}  \PY{c+c1}{\PYZsh{} 自然频率 (rad/s)}
\PY{n}{c} \PY{o}{=} \PY{l+m+mi}{2} \PY{o}{*} \PY{n}{xi} \PY{o}{*} \PY{n}{m} \PY{o}{*} \PY{n}{omega\PYZus{}n}  \PY{c+c1}{\PYZsh{} 阻尼系数 (N·s/m)}

\PY{c+c1}{\PYZsh{} 计算周期}
\PY{n}{T\PYZus{}n} \PY{o}{=} \PY{l+m+mi}{2} \PY{o}{*} \PY{n}{np}\PY{o}{.}\PY{n}{pi} \PY{o}{/} \PY{n}{omega\PYZus{}n}
\PY{n+nb}{print}\PY{p}{(}\PY{l+s+sa}{f}\PY{l+s+s2}{\PYZdq{}}\PY{l+s+s2}{系统自然周期: }\PY{l+s+si}{\PYZob{}}\PY{n}{T\PYZus{}n}\PY{l+s+si}{:}\PY{l+s+s2}{.4f}\PY{l+s+si}{\PYZcb{}}\PY{l+s+s2}{ 秒}\PY{l+s+s2}{\PYZdq{}}\PY{p}{)}

\PY{c+c1}{\PYZsh{} 外部荷载参数}
\PY{n}{F\PYZus{}0} \PY{o}{=} \PY{l+m+mf}{10.0}    \PY{c+c1}{\PYZsh{} 荷载幅值 (N)}
\PY{n}{omega\PYZus{}f} \PY{o}{=} \PY{l+m+mf}{0.8} \PY{o}{*} \PY{n}{omega\PYZus{}n}  \PY{c+c1}{\PYZsh{} 荷载频率 (rad/s)}

\PY{c+c1}{\PYZsh{} 模拟参数}
\PY{n}{t\PYZus{}final} \PY{o}{=} \PY{l+m+mi}{5} \PY{o}{*} \PY{n}{T\PYZus{}n}  \PY{c+c1}{\PYZsh{} 模拟持续时间(五个周期)}

\PY{c+c1}{\PYZsh{} 外部荷载函数}
\PY{k}{def} \PY{n+nf}{force\PYZus{}function}\PY{p}{(}\PY{n}{t}\PY{p}{)}\PY{p}{:}
    \PY{k}{return} \PY{n}{F\PYZus{}0} \PY{o}{*} \PY{n}{np}\PY{o}{.}\PY{n}{sin}\PY{p}{(}\PY{n}{omega\PYZus{}f} \PY{o}{*} \PY{n}{t}\PY{p}{)}
\end{Verbatim}
\end{tcolorbox}

    \begin{Verbatim}[commandchars=\\\{\}]
系统自然周期: 0.6283 秒
    \end{Verbatim}

    \subsection{4.
Newmark方法的实现}\label{newmarkux65b9ux6cd5ux7684ux5b9eux73b0}

下面我们实现Newmark法求解上述系统的响应:

    \begin{tcolorbox}[breakable, size=fbox, boxrule=1pt, pad at break*=1mm,colback=cellbackground, colframe=cellborder]
\prompt{In}{incolor}{3}{\boxspacing}
\begin{Verbatim}[commandchars=\\\{\}]
\PY{k}{def} \PY{n+nf}{newmark\PYZus{}method}\PY{p}{(}\PY{n}{m}\PY{p}{,} \PY{n}{c}\PY{p}{,} \PY{n}{k}\PY{p}{,} \PY{n}{p\PYZus{}func}\PY{p}{,} \PY{n}{u0}\PY{p}{,} \PY{n}{v0}\PY{p}{,} \PY{n}{t\PYZus{}final}\PY{p}{,} \PY{n}{dt}\PY{p}{,} \PY{n}{beta}\PY{o}{=}\PY{l+m+mf}{0.25}\PY{p}{,} \PY{n}{gamma}\PY{o}{=}\PY{l+m+mf}{0.5}\PY{p}{)}\PY{p}{:}
\PY{+w}{    }\PY{l+s+sd}{\PYZdq{}\PYZdq{}\PYZdq{}}
\PY{l+s+sd}{    使用Newmark\PYZhy{}β方法求解单自由度系统的动力响应}
\PY{l+s+sd}{    }
\PY{l+s+sd}{    参数:}
\PY{l+s+sd}{        m: 质量}
\PY{l+s+sd}{        c: 阻尼系数}
\PY{l+s+sd}{        k: 刚度}
\PY{l+s+sd}{        p\PYZus{}func: 外部荷载函数}
\PY{l+s+sd}{        u0: 初始位移}
\PY{l+s+sd}{        v0: 初始速度}
\PY{l+s+sd}{        t\PYZus{}final: 模拟终止时间}
\PY{l+s+sd}{        dt: 时间步长}
\PY{l+s+sd}{        beta: Newmark参数β,默认0.25(平均加速度法)}
\PY{l+s+sd}{        gamma: Newmark参数γ,默认0.5}
\PY{l+s+sd}{    }
\PY{l+s+sd}{    返回:}
\PY{l+s+sd}{        t: 时间数组}
\PY{l+s+sd}{        u: 位移数组}
\PY{l+s+sd}{        v: 速度数组}
\PY{l+s+sd}{        a: 加速度数组}
\PY{l+s+sd}{    \PYZdq{}\PYZdq{}\PYZdq{}}
    \PY{c+c1}{\PYZsh{} 计算步数}
    \PY{n}{n\PYZus{}steps} \PY{o}{=} \PY{n+nb}{int}\PY{p}{(}\PY{n}{t\PYZus{}final} \PY{o}{/} \PY{n}{dt}\PY{p}{)} \PY{o}{+} \PY{l+m+mi}{1}
    
    \PY{c+c1}{\PYZsh{} 初始化结果数组}
    \PY{n}{t} \PY{o}{=} \PY{n}{np}\PY{o}{.}\PY{n}{linspace}\PY{p}{(}\PY{l+m+mi}{0}\PY{p}{,} \PY{n}{t\PYZus{}final}\PY{p}{,} \PY{n}{n\PYZus{}steps}\PY{p}{)}
    \PY{n}{u} \PY{o}{=} \PY{n}{np}\PY{o}{.}\PY{n}{zeros}\PY{p}{(}\PY{n}{n\PYZus{}steps}\PY{p}{)}
    \PY{n}{v} \PY{o}{=} \PY{n}{np}\PY{o}{.}\PY{n}{zeros}\PY{p}{(}\PY{n}{n\PYZus{}steps}\PY{p}{)}
    \PY{n}{a} \PY{o}{=} \PY{n}{np}\PY{o}{.}\PY{n}{zeros}\PY{p}{(}\PY{n}{n\PYZus{}steps}\PY{p}{)}
    
    \PY{c+c1}{\PYZsh{} 初始条件}
    \PY{n}{u}\PY{p}{[}\PY{l+m+mi}{0}\PY{p}{]} \PY{o}{=} \PY{n}{u0}
    \PY{n}{v}\PY{p}{[}\PY{l+m+mi}{0}\PY{p}{]} \PY{o}{=} \PY{n}{v0}
    \PY{n}{a}\PY{p}{[}\PY{l+m+mi}{0}\PY{p}{]} \PY{o}{=} \PY{p}{(}\PY{n}{p\PYZus{}func}\PY{p}{(}\PY{l+m+mi}{0}\PY{p}{)} \PY{o}{\PYZhy{}} \PY{n}{c} \PY{o}{*} \PY{n}{v}\PY{p}{[}\PY{l+m+mi}{0}\PY{p}{]} \PY{o}{\PYZhy{}} \PY{n}{k} \PY{o}{*} \PY{n}{u}\PY{p}{[}\PY{l+m+mi}{0}\PY{p}{]}\PY{p}{)} \PY{o}{/} \PY{n}{m}
    
    \PY{c+c1}{\PYZsh{} Newmark方法的常数}
    \PY{n}{a1} \PY{o}{=} \PY{l+m+mf}{1.0} \PY{o}{/} \PY{p}{(}\PY{n}{beta} \PY{o}{*} \PY{n}{dt} \PY{o}{*} \PY{n}{dt}\PY{p}{)}
    \PY{n}{a2} \PY{o}{=} \PY{l+m+mf}{1.0} \PY{o}{/} \PY{p}{(}\PY{n}{beta} \PY{o}{*} \PY{n}{dt}\PY{p}{)}
    \PY{n}{a3} \PY{o}{=} \PY{p}{(}\PY{l+m+mf}{1.0} \PY{o}{\PYZhy{}} \PY{l+m+mf}{2.0} \PY{o}{*} \PY{n}{beta}\PY{p}{)} \PY{o}{/} \PY{p}{(}\PY{l+m+mf}{2.0} \PY{o}{*} \PY{n}{beta}\PY{p}{)}
    \PY{n}{a4} \PY{o}{=} \PY{n}{gamma} \PY{o}{/} \PY{p}{(}\PY{n}{beta} \PY{o}{*} \PY{n}{dt}\PY{p}{)}
    \PY{n}{a5} \PY{o}{=} \PY{l+m+mf}{1.0} \PY{o}{\PYZhy{}} \PY{n}{gamma}\PY{o}{/}\PY{n}{beta}
    \PY{n}{a6} \PY{o}{=} \PY{n}{dt} \PY{o}{*} \PY{p}{(}\PY{l+m+mf}{1.0} \PY{o}{\PYZhy{}} \PY{n}{gamma}\PY{o}{/}\PY{p}{(}\PY{l+m+mf}{2.0}\PY{o}{*}\PY{n}{beta}\PY{p}{)}\PY{p}{)}
    
    \PY{n}{k\PYZus{}hat} \PY{o}{=} \PY{n}{k} \PY{o}{+} \PY{n}{a1} \PY{o}{*} \PY{n}{m} \PY{o}{+} \PY{n}{a4} \PY{o}{*} \PY{n}{c}
    
    \PY{c+c1}{\PYZsh{} 迭代求解}
    \PY{k}{for} \PY{n}{i} \PY{o+ow}{in} \PY{n+nb}{range}\PY{p}{(}\PY{l+m+mi}{1}\PY{p}{,} \PY{n}{n\PYZus{}steps}\PY{p}{)}\PY{p}{:}
        \PY{c+c1}{\PYZsh{} 计算等效荷载}
        \PY{n}{p\PYZus{}hat} \PY{o}{=} \PY{n}{p\PYZus{}func}\PY{p}{(}\PY{n}{t}\PY{p}{[}\PY{n}{i}\PY{p}{]}\PY{p}{)} \PY{o}{+} \PY{n}{m} \PY{o}{*} \PY{p}{(}\PY{n}{a1} \PY{o}{*} \PY{n}{u}\PY{p}{[}\PY{n}{i}\PY{o}{\PYZhy{}}\PY{l+m+mi}{1}\PY{p}{]} \PY{o}{+} \PY{n}{a2} \PY{o}{*} \PY{n}{v}\PY{p}{[}\PY{n}{i}\PY{o}{\PYZhy{}}\PY{l+m+mi}{1}\PY{p}{]} \PY{o}{+} \PY{n}{a3} \PY{o}{*} \PY{n}{a}\PY{p}{[}\PY{n}{i}\PY{o}{\PYZhy{}}\PY{l+m+mi}{1}\PY{p}{]}\PY{p}{)} \PY{o}{+} \PYZbs{}
                \PY{n}{c} \PY{o}{*} \PY{p}{(}\PY{n}{a4} \PY{o}{*} \PY{n}{u}\PY{p}{[}\PY{n}{i}\PY{o}{\PYZhy{}}\PY{l+m+mi}{1}\PY{p}{]} \PY{o}{+} \PY{n}{a5} \PY{o}{*} \PY{n}{v}\PY{p}{[}\PY{n}{i}\PY{o}{\PYZhy{}}\PY{l+m+mi}{1}\PY{p}{]} \PY{o}{+} \PY{n}{a6} \PY{o}{*} \PY{n}{a}\PY{p}{[}\PY{n}{i}\PY{o}{\PYZhy{}}\PY{l+m+mi}{1}\PY{p}{]}\PY{p}{)}
        
        \PY{c+c1}{\PYZsh{} 计算当前步的位移}
        \PY{n}{u}\PY{p}{[}\PY{n}{i}\PY{p}{]} \PY{o}{=} \PY{n}{p\PYZus{}hat} \PY{o}{/} \PY{n}{k\PYZus{}hat}
        
        \PY{c+c1}{\PYZsh{} 更新加速度和速度}
        \PY{n}{a}\PY{p}{[}\PY{n}{i}\PY{p}{]} \PY{o}{=} \PY{n}{a1} \PY{o}{*} \PY{p}{(}\PY{n}{u}\PY{p}{[}\PY{n}{i}\PY{p}{]} \PY{o}{\PYZhy{}} \PY{n}{u}\PY{p}{[}\PY{n}{i}\PY{o}{\PYZhy{}}\PY{l+m+mi}{1}\PY{p}{]} \PY{o}{\PYZhy{}} \PY{n}{dt} \PY{o}{*} \PY{n}{v}\PY{p}{[}\PY{n}{i}\PY{o}{\PYZhy{}}\PY{l+m+mi}{1}\PY{p}{]}\PY{p}{)} \PY{o}{\PYZhy{}} \PY{n}{a3} \PY{o}{*} \PY{n}{a}\PY{p}{[}\PY{n}{i}\PY{o}{\PYZhy{}}\PY{l+m+mi}{1}\PY{p}{]}
        \PY{n}{v}\PY{p}{[}\PY{n}{i}\PY{p}{]} \PY{o}{=} \PY{n}{v}\PY{p}{[}\PY{n}{i}\PY{o}{\PYZhy{}}\PY{l+m+mi}{1}\PY{p}{]} \PY{o}{+} \PY{n}{dt} \PY{o}{*} \PY{p}{(}\PY{p}{(}\PY{l+m+mi}{1} \PY{o}{\PYZhy{}} \PY{n}{gamma}\PY{p}{)} \PY{o}{*} \PY{n}{a}\PY{p}{[}\PY{n}{i}\PY{o}{\PYZhy{}}\PY{l+m+mi}{1}\PY{p}{]} \PY{o}{+} \PY{n}{gamma} \PY{o}{*} \PY{n}{a}\PY{p}{[}\PY{n}{i}\PY{p}{]}\PY{p}{)}
    
    \PY{k}{return} \PY{n}{t}\PY{p}{,} \PY{n}{u}\PY{p}{,} \PY{n}{v}\PY{p}{,} \PY{n}{a}
\end{Verbatim}
\end{tcolorbox}

    \subsection{5.
不同时间步长下的精度和稳定性分析}\label{ux4e0dux540cux65f6ux95f4ux6b65ux957fux4e0bux7684ux7cbeux5ea6ux548cux7a33ux5b9aux6027ux5206ux6790}

现在我们使用不同的时间步长来求解系统响应,并比较结果。我们将设置三种不同的时间步长来观察Newmark方法的性能。

    \begin{tcolorbox}[breakable, size=fbox, boxrule=1pt, pad at break*=1mm,colback=cellbackground, colframe=cellborder]
\prompt{In}{incolor}{4}{\boxspacing}
\begin{Verbatim}[commandchars=\\\{\}]
\PY{c+c1}{\PYZsh{} 设置初始条件}
\PY{n}{u0} \PY{o}{=} \PY{l+m+mf}{0.0}  \PY{c+c1}{\PYZsh{} 初始位移}
\PY{n}{v0} \PY{o}{=} \PY{l+m+mf}{0.0}  \PY{c+c1}{\PYZsh{} 初始速度}

\PY{c+c1}{\PYZsh{} 创建不同的时间步长}
\PY{n}{dt1} \PY{o}{=} \PY{n}{T\PYZus{}n} \PY{o}{/} \PY{l+m+mi}{50}  \PY{c+c1}{\PYZsh{} 较小的时间步长}
\PY{n}{dt2} \PY{o}{=} \PY{n}{T\PYZus{}n} \PY{o}{/} \PY{l+m+mi}{20}  \PY{c+c1}{\PYZsh{} 中等时间步长}
\PY{n}{dt3} \PY{o}{=} \PY{n}{T\PYZus{}n} \PY{o}{/} \PY{l+m+mi}{5}   \PY{c+c1}{\PYZsh{} 较大的时间步长}

\PY{n+nb}{print}\PY{p}{(}\PY{l+s+sa}{f}\PY{l+s+s2}{\PYZdq{}}\PY{l+s+s2}{时间步长设置:}\PY{l+s+s2}{\PYZdq{}}\PY{p}{)}
\PY{n+nb}{print}\PY{p}{(}\PY{l+s+sa}{f}\PY{l+s+s2}{\PYZdq{}}\PY{l+s+s2}{dt1 = T\PYZus{}n/50 = }\PY{l+s+si}{\PYZob{}}\PY{n}{dt1}\PY{l+s+si}{:}\PY{l+s+s2}{.6f}\PY{l+s+si}{\PYZcb{}}\PY{l+s+s2}{ 秒}\PY{l+s+s2}{\PYZdq{}}\PY{p}{)}
\PY{n+nb}{print}\PY{p}{(}\PY{l+s+sa}{f}\PY{l+s+s2}{\PYZdq{}}\PY{l+s+s2}{dt2 = T\PYZus{}n/20 = }\PY{l+s+si}{\PYZob{}}\PY{n}{dt2}\PY{l+s+si}{:}\PY{l+s+s2}{.6f}\PY{l+s+si}{\PYZcb{}}\PY{l+s+s2}{ 秒}\PY{l+s+s2}{\PYZdq{}}\PY{p}{)}
\PY{n+nb}{print}\PY{p}{(}\PY{l+s+sa}{f}\PY{l+s+s2}{\PYZdq{}}\PY{l+s+s2}{dt3 = T\PYZus{}n/5 = }\PY{l+s+si}{\PYZob{}}\PY{n}{dt3}\PY{l+s+si}{:}\PY{l+s+s2}{.6f}\PY{l+s+si}{\PYZcb{}}\PY{l+s+s2}{ 秒}\PY{l+s+s2}{\PYZdq{}}\PY{p}{)}
\end{Verbatim}
\end{tcolorbox}

    \begin{Verbatim}[commandchars=\\\{\}]
时间步长设置:
dt1 = T\_n/50 = 0.012566 秒
dt2 = T\_n/20 = 0.031416 秒
dt3 = T\_n/5 = 0.125664 秒
    \end{Verbatim}

    \begin{tcolorbox}[breakable, size=fbox, boxrule=1pt, pad at break*=1mm,colback=cellbackground, colframe=cellborder]
\prompt{In}{incolor}{5}{\boxspacing}
\begin{Verbatim}[commandchars=\\\{\}]
\PY{c+c1}{\PYZsh{} 平均加速度法 (β=0.25, γ=0.5)}
\PY{n}{beta\PYZus{}avg} \PY{o}{=} \PY{l+m+mf}{0.25}
\PY{n}{gamma\PYZus{}avg} \PY{o}{=} \PY{l+m+mf}{0.5}

\PY{c+c1}{\PYZsh{} 使用不同时间步长进行计算}
\PY{n}{t1}\PY{p}{,} \PY{n}{u1}\PY{p}{,} \PY{n}{v1}\PY{p}{,} \PY{n}{a1} \PY{o}{=} \PY{n}{newmark\PYZus{}method}\PY{p}{(}\PY{n}{m}\PY{p}{,} \PY{n}{c}\PY{p}{,} \PY{n}{k}\PY{p}{,} \PY{n}{force\PYZus{}function}\PY{p}{,} \PY{n}{u0}\PY{p}{,} \PY{n}{v0}\PY{p}{,} \PY{n}{t\PYZus{}final}\PY{p}{,} \PY{n}{dt1}\PY{p}{,} \PY{n}{beta\PYZus{}avg}\PY{p}{,} \PY{n}{gamma\PYZus{}avg}\PY{p}{)}
\PY{n}{t2}\PY{p}{,} \PY{n}{u2}\PY{p}{,} \PY{n}{v2}\PY{p}{,} \PY{n}{a2} \PY{o}{=} \PY{n}{newmark\PYZus{}method}\PY{p}{(}\PY{n}{m}\PY{p}{,} \PY{n}{c}\PY{p}{,} \PY{n}{k}\PY{p}{,} \PY{n}{force\PYZus{}function}\PY{p}{,} \PY{n}{u0}\PY{p}{,} \PY{n}{v0}\PY{p}{,} \PY{n}{t\PYZus{}final}\PY{p}{,} \PY{n}{dt2}\PY{p}{,} \PY{n}{beta\PYZus{}avg}\PY{p}{,} \PY{n}{gamma\PYZus{}avg}\PY{p}{)}
\PY{n}{t3}\PY{p}{,} \PY{n}{u3}\PY{p}{,} \PY{n}{v3}\PY{p}{,} \PY{n}{a3} \PY{o}{=} \PY{n}{newmark\PYZus{}method}\PY{p}{(}\PY{n}{m}\PY{p}{,} \PY{n}{c}\PY{p}{,} \PY{n}{k}\PY{p}{,} \PY{n}{force\PYZus{}function}\PY{p}{,} \PY{n}{u0}\PY{p}{,} \PY{n}{v0}\PY{p}{,} \PY{n}{t\PYZus{}final}\PY{p}{,} \PY{n}{dt3}\PY{p}{,} \PY{n}{beta\PYZus{}avg}\PY{p}{,} \PY{n}{gamma\PYZus{}avg}\PY{p}{)}

\PY{c+c1}{\PYZsh{} 对比可视化}
\PY{n}{plt}\PY{o}{.}\PY{n}{figure}\PY{p}{(}\PY{n}{figsize}\PY{o}{=}\PY{p}{(}\PY{l+m+mi}{14}\PY{p}{,} \PY{l+m+mi}{10}\PY{p}{)}\PY{p}{)}

\PY{c+c1}{\PYZsh{} 位移对比}
\PY{n}{plt}\PY{o}{.}\PY{n}{subplot}\PY{p}{(}\PY{l+m+mi}{3}\PY{p}{,} \PY{l+m+mi}{1}\PY{p}{,} \PY{l+m+mi}{1}\PY{p}{)}
\PY{n}{plt}\PY{o}{.}\PY{n}{plot}\PY{p}{(}\PY{n}{t1}\PY{p}{,} \PY{n}{u1}\PY{p}{,} \PY{l+s+s1}{\PYZsq{}}\PY{l+s+s1}{b\PYZhy{}}\PY{l+s+s1}{\PYZsq{}}\PY{p}{,} \PY{n}{label}\PY{o}{=}\PY{l+s+sa}{f}\PY{l+s+s1}{\PYZsq{}}\PY{l+s+s1}{dt = T\PYZus{}n/50}\PY{l+s+s1}{\PYZsq{}}\PY{p}{)}
\PY{n}{plt}\PY{o}{.}\PY{n}{plot}\PY{p}{(}\PY{n}{t2}\PY{p}{,} \PY{n}{u2}\PY{p}{,} \PY{l+s+s1}{\PYZsq{}}\PY{l+s+s1}{g\PYZhy{}\PYZhy{}}\PY{l+s+s1}{\PYZsq{}}\PY{p}{,} \PY{n}{label}\PY{o}{=}\PY{l+s+sa}{f}\PY{l+s+s1}{\PYZsq{}}\PY{l+s+s1}{dt = T\PYZus{}n/20}\PY{l+s+s1}{\PYZsq{}}\PY{p}{)}
\PY{n}{plt}\PY{o}{.}\PY{n}{plot}\PY{p}{(}\PY{n}{t3}\PY{p}{,} \PY{n}{u3}\PY{p}{,} \PY{l+s+s1}{\PYZsq{}}\PY{l+s+s1}{r\PYZhy{}.}\PY{l+s+s1}{\PYZsq{}}\PY{p}{,} \PY{n}{label}\PY{o}{=}\PY{l+s+sa}{f}\PY{l+s+s1}{\PYZsq{}}\PY{l+s+s1}{dt = T\PYZus{}n/5}\PY{l+s+s1}{\PYZsq{}}\PY{p}{)}
\PY{n}{plt}\PY{o}{.}\PY{n}{xlabel}\PY{p}{(}\PY{l+s+s1}{\PYZsq{}}\PY{l+s+s1}{时间 (秒)}\PY{l+s+s1}{\PYZsq{}}\PY{p}{)}
\PY{n}{plt}\PY{o}{.}\PY{n}{ylabel}\PY{p}{(}\PY{l+s+s1}{\PYZsq{}}\PY{l+s+s1}{位移 (m)}\PY{l+s+s1}{\PYZsq{}}\PY{p}{)}
\PY{n}{plt}\PY{o}{.}\PY{n}{title}\PY{p}{(}\PY{l+s+s1}{\PYZsq{}}\PY{l+s+s1}{不同时间步长下的位移响应对比 (平均加速度法)}\PY{l+s+s1}{\PYZsq{}}\PY{p}{)}
\PY{n}{plt}\PY{o}{.}\PY{n}{legend}\PY{p}{(}\PY{p}{)}
\PY{n}{plt}\PY{o}{.}\PY{n}{grid}\PY{p}{(}\PY{k+kc}{True}\PY{p}{)}

\PY{c+c1}{\PYZsh{} 速度对比}
\PY{n}{plt}\PY{o}{.}\PY{n}{subplot}\PY{p}{(}\PY{l+m+mi}{3}\PY{p}{,} \PY{l+m+mi}{1}\PY{p}{,} \PY{l+m+mi}{2}\PY{p}{)}
\PY{n}{plt}\PY{o}{.}\PY{n}{plot}\PY{p}{(}\PY{n}{t1}\PY{p}{,} \PY{n}{v1}\PY{p}{,} \PY{l+s+s1}{\PYZsq{}}\PY{l+s+s1}{b\PYZhy{}}\PY{l+s+s1}{\PYZsq{}}\PY{p}{,} \PY{n}{label}\PY{o}{=}\PY{l+s+sa}{f}\PY{l+s+s1}{\PYZsq{}}\PY{l+s+s1}{dt = T\PYZus{}n/50}\PY{l+s+s1}{\PYZsq{}}\PY{p}{)}
\PY{n}{plt}\PY{o}{.}\PY{n}{plot}\PY{p}{(}\PY{n}{t2}\PY{p}{,} \PY{n}{v2}\PY{p}{,} \PY{l+s+s1}{\PYZsq{}}\PY{l+s+s1}{g\PYZhy{}\PYZhy{}}\PY{l+s+s1}{\PYZsq{}}\PY{p}{,} \PY{n}{label}\PY{o}{=}\PY{l+s+sa}{f}\PY{l+s+s1}{\PYZsq{}}\PY{l+s+s1}{dt = T\PYZus{}n/20}\PY{l+s+s1}{\PYZsq{}}\PY{p}{)}
\PY{n}{plt}\PY{o}{.}\PY{n}{plot}\PY{p}{(}\PY{n}{t3}\PY{p}{,} \PY{n}{v3}\PY{p}{,} \PY{l+s+s1}{\PYZsq{}}\PY{l+s+s1}{r\PYZhy{}.}\PY{l+s+s1}{\PYZsq{}}\PY{p}{,} \PY{n}{label}\PY{o}{=}\PY{l+s+sa}{f}\PY{l+s+s1}{\PYZsq{}}\PY{l+s+s1}{dt = T\PYZus{}n/5}\PY{l+s+s1}{\PYZsq{}}\PY{p}{)}
\PY{n}{plt}\PY{o}{.}\PY{n}{xlabel}\PY{p}{(}\PY{l+s+s1}{\PYZsq{}}\PY{l+s+s1}{时间 (秒)}\PY{l+s+s1}{\PYZsq{}}\PY{p}{)}
\PY{n}{plt}\PY{o}{.}\PY{n}{ylabel}\PY{p}{(}\PY{l+s+s1}{\PYZsq{}}\PY{l+s+s1}{速度 (m/s)}\PY{l+s+s1}{\PYZsq{}}\PY{p}{)}
\PY{n}{plt}\PY{o}{.}\PY{n}{title}\PY{p}{(}\PY{l+s+s1}{\PYZsq{}}\PY{l+s+s1}{不同时间步长下的速度响应对比 (平均加速度法)}\PY{l+s+s1}{\PYZsq{}}\PY{p}{)}
\PY{n}{plt}\PY{o}{.}\PY{n}{legend}\PY{p}{(}\PY{p}{)}
\PY{n}{plt}\PY{o}{.}\PY{n}{grid}\PY{p}{(}\PY{k+kc}{True}\PY{p}{)}

\PY{c+c1}{\PYZsh{} 加速度对比}
\PY{n}{plt}\PY{o}{.}\PY{n}{subplot}\PY{p}{(}\PY{l+m+mi}{3}\PY{p}{,} \PY{l+m+mi}{1}\PY{p}{,} \PY{l+m+mi}{3}\PY{p}{)}
\PY{n}{plt}\PY{o}{.}\PY{n}{plot}\PY{p}{(}\PY{n}{t1}\PY{p}{,} \PY{n}{a1}\PY{p}{,} \PY{l+s+s1}{\PYZsq{}}\PY{l+s+s1}{b\PYZhy{}}\PY{l+s+s1}{\PYZsq{}}\PY{p}{,} \PY{n}{label}\PY{o}{=}\PY{l+s+sa}{f}\PY{l+s+s1}{\PYZsq{}}\PY{l+s+s1}{dt = T\PYZus{}n/50}\PY{l+s+s1}{\PYZsq{}}\PY{p}{)}
\PY{n}{plt}\PY{o}{.}\PY{n}{plot}\PY{p}{(}\PY{n}{t2}\PY{p}{,} \PY{n}{a2}\PY{p}{,} \PY{l+s+s1}{\PYZsq{}}\PY{l+s+s1}{g\PYZhy{}\PYZhy{}}\PY{l+s+s1}{\PYZsq{}}\PY{p}{,} \PY{n}{label}\PY{o}{=}\PY{l+s+sa}{f}\PY{l+s+s1}{\PYZsq{}}\PY{l+s+s1}{dt = T\PYZus{}n/20}\PY{l+s+s1}{\PYZsq{}}\PY{p}{)}
\PY{n}{plt}\PY{o}{.}\PY{n}{plot}\PY{p}{(}\PY{n}{t3}\PY{p}{,} \PY{n}{a3}\PY{p}{,} \PY{l+s+s1}{\PYZsq{}}\PY{l+s+s1}{r\PYZhy{}.}\PY{l+s+s1}{\PYZsq{}}\PY{p}{,} \PY{n}{label}\PY{o}{=}\PY{l+s+sa}{f}\PY{l+s+s1}{\PYZsq{}}\PY{l+s+s1}{dt = T\PYZus{}n/5}\PY{l+s+s1}{\PYZsq{}}\PY{p}{)}
\PY{n}{plt}\PY{o}{.}\PY{n}{xlabel}\PY{p}{(}\PY{l+s+s1}{\PYZsq{}}\PY{l+s+s1}{时间 (秒)}\PY{l+s+s1}{\PYZsq{}}\PY{p}{)}
\PY{n}{plt}\PY{o}{.}\PY{n}{ylabel}\PY{p}{(}\PY{l+s+s1}{\PYZsq{}}\PY{l+s+s1}{加速度 (m/s²)}\PY{l+s+s1}{\PYZsq{}}\PY{p}{)}
\PY{n}{plt}\PY{o}{.}\PY{n}{title}\PY{p}{(}\PY{l+s+s1}{\PYZsq{}}\PY{l+s+s1}{不同时间步长下的加速度响应对比 (平均加速度法)}\PY{l+s+s1}{\PYZsq{}}\PY{p}{)}
\PY{n}{plt}\PY{o}{.}\PY{n}{legend}\PY{p}{(}\PY{p}{)}
\PY{n}{plt}\PY{o}{.}\PY{n}{grid}\PY{p}{(}\PY{k+kc}{True}\PY{p}{)}

\PY{n}{plt}\PY{o}{.}\PY{n}{tight\PYZus{}layout}\PY{p}{(}\PY{p}{)}
\PY{n}{plt}\PY{o}{.}\PY{n}{show}\PY{p}{(}\PY{p}{)}
\end{Verbatim}
\end{tcolorbox}

    \begin{Verbatim}[commandchars=\\\{\}]
/var/folders/tm/snrp8gcn00113mpkf\_j5jmqm0000gn/T/ipykernel\_8490/11501908.py:46:
UserWarning: Glyph 26102 (\textbackslash{}N\{CJK UNIFIED IDEOGRAPH-65F6\}) missing from current
font.
  plt.tight\_layout()
/var/folders/tm/snrp8gcn00113mpkf\_j5jmqm0000gn/T/ipykernel\_8490/11501908.py:46:
UserWarning: Glyph 38388 (\textbackslash{}N\{CJK UNIFIED IDEOGRAPH-95F4\}) missing from current
font.
  plt.tight\_layout()
/var/folders/tm/snrp8gcn00113mpkf\_j5jmqm0000gn/T/ipykernel\_8490/11501908.py:46:
UserWarning: Glyph 31186 (\textbackslash{}N\{CJK UNIFIED IDEOGRAPH-79D2\}) missing from current
font.
  plt.tight\_layout()
/var/folders/tm/snrp8gcn00113mpkf\_j5jmqm0000gn/T/ipykernel\_8490/11501908.py:46:
UserWarning: Glyph 20301 (\textbackslash{}N\{CJK UNIFIED IDEOGRAPH-4F4D\}) missing from current
font.
  plt.tight\_layout()
/var/folders/tm/snrp8gcn00113mpkf\_j5jmqm0000gn/T/ipykernel\_8490/11501908.py:46:
UserWarning: Glyph 31227 (\textbackslash{}N\{CJK UNIFIED IDEOGRAPH-79FB\}) missing from current
font.
  plt.tight\_layout()
/var/folders/tm/snrp8gcn00113mpkf\_j5jmqm0000gn/T/ipykernel\_8490/11501908.py:46:
UserWarning: Glyph 19981 (\textbackslash{}N\{CJK UNIFIED IDEOGRAPH-4E0D\}) missing from current
font.
  plt.tight\_layout()
/var/folders/tm/snrp8gcn00113mpkf\_j5jmqm0000gn/T/ipykernel\_8490/11501908.py:46:
UserWarning: Glyph 21516 (\textbackslash{}N\{CJK UNIFIED IDEOGRAPH-540C\}) missing from current
font.
  plt.tight\_layout()
/var/folders/tm/snrp8gcn00113mpkf\_j5jmqm0000gn/T/ipykernel\_8490/11501908.py:46:
UserWarning: Glyph 27493 (\textbackslash{}N\{CJK UNIFIED IDEOGRAPH-6B65\}) missing from current
font.
  plt.tight\_layout()
/var/folders/tm/snrp8gcn00113mpkf\_j5jmqm0000gn/T/ipykernel\_8490/11501908.py:46:
UserWarning: Glyph 38271 (\textbackslash{}N\{CJK UNIFIED IDEOGRAPH-957F\}) missing from current
font.
  plt.tight\_layout()
/var/folders/tm/snrp8gcn00113mpkf\_j5jmqm0000gn/T/ipykernel\_8490/11501908.py:46:
UserWarning: Glyph 19979 (\textbackslash{}N\{CJK UNIFIED IDEOGRAPH-4E0B\}) missing from current
font.
  plt.tight\_layout()
/var/folders/tm/snrp8gcn00113mpkf\_j5jmqm0000gn/T/ipykernel\_8490/11501908.py:46:
UserWarning: Glyph 30340 (\textbackslash{}N\{CJK UNIFIED IDEOGRAPH-7684\}) missing from current
font.
  plt.tight\_layout()
/var/folders/tm/snrp8gcn00113mpkf\_j5jmqm0000gn/T/ipykernel\_8490/11501908.py:46:
UserWarning: Glyph 21709 (\textbackslash{}N\{CJK UNIFIED IDEOGRAPH-54CD\}) missing from current
font.
  plt.tight\_layout()
/var/folders/tm/snrp8gcn00113mpkf\_j5jmqm0000gn/T/ipykernel\_8490/11501908.py:46:
UserWarning: Glyph 24212 (\textbackslash{}N\{CJK UNIFIED IDEOGRAPH-5E94\}) missing from current
font.
  plt.tight\_layout()
/var/folders/tm/snrp8gcn00113mpkf\_j5jmqm0000gn/T/ipykernel\_8490/11501908.py:46:
UserWarning: Glyph 23545 (\textbackslash{}N\{CJK UNIFIED IDEOGRAPH-5BF9\}) missing from current
font.
  plt.tight\_layout()
/var/folders/tm/snrp8gcn00113mpkf\_j5jmqm0000gn/T/ipykernel\_8490/11501908.py:46:
UserWarning: Glyph 27604 (\textbackslash{}N\{CJK UNIFIED IDEOGRAPH-6BD4\}) missing from current
font.
  plt.tight\_layout()
/var/folders/tm/snrp8gcn00113mpkf\_j5jmqm0000gn/T/ipykernel\_8490/11501908.py:46:
UserWarning: Glyph 24179 (\textbackslash{}N\{CJK UNIFIED IDEOGRAPH-5E73\}) missing from current
font.
  plt.tight\_layout()
/var/folders/tm/snrp8gcn00113mpkf\_j5jmqm0000gn/T/ipykernel\_8490/11501908.py:46:
UserWarning: Glyph 22343 (\textbackslash{}N\{CJK UNIFIED IDEOGRAPH-5747\}) missing from current
font.
  plt.tight\_layout()
/var/folders/tm/snrp8gcn00113mpkf\_j5jmqm0000gn/T/ipykernel\_8490/11501908.py:46:
UserWarning: Glyph 21152 (\textbackslash{}N\{CJK UNIFIED IDEOGRAPH-52A0\}) missing from current
font.
  plt.tight\_layout()
/var/folders/tm/snrp8gcn00113mpkf\_j5jmqm0000gn/T/ipykernel\_8490/11501908.py:46:
UserWarning: Glyph 36895 (\textbackslash{}N\{CJK UNIFIED IDEOGRAPH-901F\}) missing from current
font.
  plt.tight\_layout()
/var/folders/tm/snrp8gcn00113mpkf\_j5jmqm0000gn/T/ipykernel\_8490/11501908.py:46:
UserWarning: Glyph 24230 (\textbackslash{}N\{CJK UNIFIED IDEOGRAPH-5EA6\}) missing from current
font.
  plt.tight\_layout()
/var/folders/tm/snrp8gcn00113mpkf\_j5jmqm0000gn/T/ipykernel\_8490/11501908.py:46:
UserWarning: Glyph 27861 (\textbackslash{}N\{CJK UNIFIED IDEOGRAPH-6CD5\}) missing from current
font.
  plt.tight\_layout()
/Users/henri/miniconda3/envs/henri\_env/lib/python3.10/site-
packages/IPython/core/pylabtools.py:152: UserWarning: Glyph 20301 (\textbackslash{}N\{CJK
UNIFIED IDEOGRAPH-4F4D\}) missing from current font.
  fig.canvas.print\_figure(bytes\_io, **kw)
/Users/henri/miniconda3/envs/henri\_env/lib/python3.10/site-
packages/IPython/core/pylabtools.py:152: UserWarning: Glyph 31227 (\textbackslash{}N\{CJK
UNIFIED IDEOGRAPH-79FB\}) missing from current font.
  fig.canvas.print\_figure(bytes\_io, **kw)
/Users/henri/miniconda3/envs/henri\_env/lib/python3.10/site-
packages/IPython/core/pylabtools.py:152: UserWarning: Glyph 19981 (\textbackslash{}N\{CJK
UNIFIED IDEOGRAPH-4E0D\}) missing from current font.
  fig.canvas.print\_figure(bytes\_io, **kw)
/Users/henri/miniconda3/envs/henri\_env/lib/python3.10/site-
packages/IPython/core/pylabtools.py:152: UserWarning: Glyph 21516 (\textbackslash{}N\{CJK
UNIFIED IDEOGRAPH-540C\}) missing from current font.
  fig.canvas.print\_figure(bytes\_io, **kw)
/Users/henri/miniconda3/envs/henri\_env/lib/python3.10/site-
packages/IPython/core/pylabtools.py:152: UserWarning: Glyph 26102 (\textbackslash{}N\{CJK
UNIFIED IDEOGRAPH-65F6\}) missing from current font.
  fig.canvas.print\_figure(bytes\_io, **kw)
/Users/henri/miniconda3/envs/henri\_env/lib/python3.10/site-
packages/IPython/core/pylabtools.py:152: UserWarning: Glyph 38388 (\textbackslash{}N\{CJK
UNIFIED IDEOGRAPH-95F4\}) missing from current font.
  fig.canvas.print\_figure(bytes\_io, **kw)
/Users/henri/miniconda3/envs/henri\_env/lib/python3.10/site-
packages/IPython/core/pylabtools.py:152: UserWarning: Glyph 27493 (\textbackslash{}N\{CJK
UNIFIED IDEOGRAPH-6B65\}) missing from current font.
  fig.canvas.print\_figure(bytes\_io, **kw)
/Users/henri/miniconda3/envs/henri\_env/lib/python3.10/site-
packages/IPython/core/pylabtools.py:152: UserWarning: Glyph 38271 (\textbackslash{}N\{CJK
UNIFIED IDEOGRAPH-957F\}) missing from current font.
  fig.canvas.print\_figure(bytes\_io, **kw)
/Users/henri/miniconda3/envs/henri\_env/lib/python3.10/site-
packages/IPython/core/pylabtools.py:152: UserWarning: Glyph 19979 (\textbackslash{}N\{CJK
UNIFIED IDEOGRAPH-4E0B\}) missing from current font.
  fig.canvas.print\_figure(bytes\_io, **kw)
/Users/henri/miniconda3/envs/henri\_env/lib/python3.10/site-
packages/IPython/core/pylabtools.py:152: UserWarning: Glyph 30340 (\textbackslash{}N\{CJK
UNIFIED IDEOGRAPH-7684\}) missing from current font.
  fig.canvas.print\_figure(bytes\_io, **kw)
/Users/henri/miniconda3/envs/henri\_env/lib/python3.10/site-
packages/IPython/core/pylabtools.py:152: UserWarning: Glyph 21709 (\textbackslash{}N\{CJK
UNIFIED IDEOGRAPH-54CD\}) missing from current font.
  fig.canvas.print\_figure(bytes\_io, **kw)
/Users/henri/miniconda3/envs/henri\_env/lib/python3.10/site-
packages/IPython/core/pylabtools.py:152: UserWarning: Glyph 24212 (\textbackslash{}N\{CJK
UNIFIED IDEOGRAPH-5E94\}) missing from current font.
  fig.canvas.print\_figure(bytes\_io, **kw)
/Users/henri/miniconda3/envs/henri\_env/lib/python3.10/site-
packages/IPython/core/pylabtools.py:152: UserWarning: Glyph 23545 (\textbackslash{}N\{CJK
UNIFIED IDEOGRAPH-5BF9\}) missing from current font.
  fig.canvas.print\_figure(bytes\_io, **kw)
/Users/henri/miniconda3/envs/henri\_env/lib/python3.10/site-
packages/IPython/core/pylabtools.py:152: UserWarning: Glyph 27604 (\textbackslash{}N\{CJK
UNIFIED IDEOGRAPH-6BD4\}) missing from current font.
  fig.canvas.print\_figure(bytes\_io, **kw)
/Users/henri/miniconda3/envs/henri\_env/lib/python3.10/site-
packages/IPython/core/pylabtools.py:152: UserWarning: Glyph 24179 (\textbackslash{}N\{CJK
UNIFIED IDEOGRAPH-5E73\}) missing from current font.
  fig.canvas.print\_figure(bytes\_io, **kw)
/Users/henri/miniconda3/envs/henri\_env/lib/python3.10/site-
packages/IPython/core/pylabtools.py:152: UserWarning: Glyph 22343 (\textbackslash{}N\{CJK
UNIFIED IDEOGRAPH-5747\}) missing from current font.
  fig.canvas.print\_figure(bytes\_io, **kw)
/Users/henri/miniconda3/envs/henri\_env/lib/python3.10/site-
packages/IPython/core/pylabtools.py:152: UserWarning: Glyph 21152 (\textbackslash{}N\{CJK
UNIFIED IDEOGRAPH-52A0\}) missing from current font.
  fig.canvas.print\_figure(bytes\_io, **kw)
/Users/henri/miniconda3/envs/henri\_env/lib/python3.10/site-
packages/IPython/core/pylabtools.py:152: UserWarning: Glyph 36895 (\textbackslash{}N\{CJK
UNIFIED IDEOGRAPH-901F\}) missing from current font.
  fig.canvas.print\_figure(bytes\_io, **kw)
/Users/henri/miniconda3/envs/henri\_env/lib/python3.10/site-
packages/IPython/core/pylabtools.py:152: UserWarning: Glyph 24230 (\textbackslash{}N\{CJK
UNIFIED IDEOGRAPH-5EA6\}) missing from current font.
  fig.canvas.print\_figure(bytes\_io, **kw)
/Users/henri/miniconda3/envs/henri\_env/lib/python3.10/site-
packages/IPython/core/pylabtools.py:152: UserWarning: Glyph 27861 (\textbackslash{}N\{CJK
UNIFIED IDEOGRAPH-6CD5\}) missing from current font.
  fig.canvas.print\_figure(bytes\_io, **kw)
/Users/henri/miniconda3/envs/henri\_env/lib/python3.10/site-
packages/IPython/core/pylabtools.py:152: UserWarning: Glyph 31186 (\textbackslash{}N\{CJK
UNIFIED IDEOGRAPH-79D2\}) missing from current font.
  fig.canvas.print\_figure(bytes\_io, **kw)
    \end{Verbatim}

    \begin{center}
    \adjustimage{max size={0.9\linewidth}{0.9\paperheight}}{output_10_1.png}
    \end{center}
    { \hspace*{\fill} \\}
    
    \subsection{6.
不同Newmark参数的稳定性对比}\label{ux4e0dux540cnewmarkux53c2ux6570ux7684ux7a33ux5b9aux6027ux5bf9ux6bd4}

现在我们比较不同的Newmark参数设置对结果的影响。主要比较平均加速度法与线性加速度法的区别。

    \begin{tcolorbox}[breakable, size=fbox, boxrule=1pt, pad at break*=1mm,colback=cellbackground, colframe=cellborder]
\prompt{In}{incolor}{6}{\boxspacing}
\begin{Verbatim}[commandchars=\\\{\}]
\PY{c+c1}{\PYZsh{} 线性加速度法参数 (β=1/6, γ=1/2)}
\PY{n}{beta\PYZus{}linear} \PY{o}{=} \PY{l+m+mi}{1}\PY{o}{/}\PY{l+m+mi}{6}
\PY{n}{gamma\PYZus{}linear} \PY{o}{=} \PY{l+m+mf}{0.5}

\PY{c+c1}{\PYZsh{} 使用较大时间步长计算,以观察稳定性差异}
\PY{n}{dt\PYZus{}large} \PY{o}{=} \PY{n}{T\PYZus{}n} \PY{o}{/} \PY{l+m+mi}{3}  \PY{c+c1}{\PYZsh{} 一个较大的时间步长,可能会导致线性加速度法不稳定}

\PY{c+c1}{\PYZsh{} 平均加速度法(无条件稳定)}
\PY{n}{t\PYZus{}avg}\PY{p}{,} \PY{n}{u\PYZus{}avg}\PY{p}{,} \PY{n}{v\PYZus{}avg}\PY{p}{,} \PY{n}{a\PYZus{}avg} \PY{o}{=} \PY{n}{newmark\PYZus{}method}\PY{p}{(}\PY{n}{m}\PY{p}{,} \PY{n}{c}\PY{p}{,} \PY{n}{k}\PY{p}{,} \PY{n}{force\PYZus{}function}\PY{p}{,} \PY{n}{u0}\PY{p}{,} \PY{n}{v0}\PY{p}{,} \PY{n}{t\PYZus{}final}\PY{p}{,} \PY{n}{dt\PYZus{}large}\PY{p}{,} \PY{n}{beta\PYZus{}avg}\PY{p}{,} \PY{n}{gamma\PYZus{}avg}\PY{p}{)}

\PY{c+c1}{\PYZsh{} 线性加速度法(条件稳定)}
\PY{n}{t\PYZus{}lin}\PY{p}{,} \PY{n}{u\PYZus{}lin}\PY{p}{,} \PY{n}{v\PYZus{}lin}\PY{p}{,} \PY{n}{a\PYZus{}lin} \PY{o}{=} \PY{n}{newmark\PYZus{}method}\PY{p}{(}\PY{n}{m}\PY{p}{,} \PY{n}{c}\PY{p}{,} \PY{n}{k}\PY{p}{,} \PY{n}{force\PYZus{}function}\PY{p}{,} \PY{n}{u0}\PY{p}{,} \PY{n}{v0}\PY{p}{,} \PY{n}{t\PYZus{}final}\PY{p}{,} \PY{n}{dt\PYZus{}large}\PY{p}{,} \PY{n}{beta\PYZus{}linear}\PY{p}{,} \PY{n}{gamma\PYZus{}linear}\PY{p}{)}

\PY{c+c1}{\PYZsh{} 对比可视化}
\PY{n}{plt}\PY{o}{.}\PY{n}{figure}\PY{p}{(}\PY{n}{figsize}\PY{o}{=}\PY{p}{(}\PY{l+m+mi}{14}\PY{p}{,} \PY{l+m+mi}{8}\PY{p}{)}\PY{p}{)}

\PY{n}{plt}\PY{o}{.}\PY{n}{subplot}\PY{p}{(}\PY{l+m+mi}{2}\PY{p}{,} \PY{l+m+mi}{1}\PY{p}{,} \PY{l+m+mi}{1}\PY{p}{)}
\PY{n}{plt}\PY{o}{.}\PY{n}{plot}\PY{p}{(}\PY{n}{t\PYZus{}avg}\PY{p}{,} \PY{n}{u\PYZus{}avg}\PY{p}{,} \PY{l+s+s1}{\PYZsq{}}\PY{l+s+s1}{b\PYZhy{}}\PY{l+s+s1}{\PYZsq{}}\PY{p}{,} \PY{n}{label}\PY{o}{=}\PY{l+s+s1}{\PYZsq{}}\PY{l+s+s1}{平均加速度法 (β=1/4)}\PY{l+s+s1}{\PYZsq{}}\PY{p}{)}
\PY{n}{plt}\PY{o}{.}\PY{n}{plot}\PY{p}{(}\PY{n}{t\PYZus{}lin}\PY{p}{,} \PY{n}{u\PYZus{}lin}\PY{p}{,} \PY{l+s+s1}{\PYZsq{}}\PY{l+s+s1}{r\PYZhy{}\PYZhy{}}\PY{l+s+s1}{\PYZsq{}}\PY{p}{,} \PY{n}{label}\PY{o}{=}\PY{l+s+s1}{\PYZsq{}}\PY{l+s+s1}{线性加速度法 (β=1/6)}\PY{l+s+s1}{\PYZsq{}}\PY{p}{)}
\PY{n}{plt}\PY{o}{.}\PY{n}{xlabel}\PY{p}{(}\PY{l+s+s1}{\PYZsq{}}\PY{l+s+s1}{时间 (秒)}\PY{l+s+s1}{\PYZsq{}}\PY{p}{)}
\PY{n}{plt}\PY{o}{.}\PY{n}{ylabel}\PY{p}{(}\PY{l+s+s1}{\PYZsq{}}\PY{l+s+s1}{位移 (m)}\PY{l+s+s1}{\PYZsq{}}\PY{p}{)}
\PY{n}{plt}\PY{o}{.}\PY{n}{title}\PY{p}{(}\PY{l+s+sa}{f}\PY{l+s+s1}{\PYZsq{}}\PY{l+s+s1}{不同Newmark参数下的位移响应对比 (dt = }\PY{l+s+si}{\PYZob{}}\PY{n}{dt\PYZus{}large}\PY{l+s+si}{:}\PY{l+s+s1}{.4f}\PY{l+s+si}{\PYZcb{}}\PY{l+s+s1}{秒)}\PY{l+s+s1}{\PYZsq{}}\PY{p}{)}
\PY{n}{plt}\PY{o}{.}\PY{n}{legend}\PY{p}{(}\PY{p}{)}
\PY{n}{plt}\PY{o}{.}\PY{n}{grid}\PY{p}{(}\PY{k+kc}{True}\PY{p}{)}

\PY{n}{plt}\PY{o}{.}\PY{n}{subplot}\PY{p}{(}\PY{l+m+mi}{2}\PY{p}{,} \PY{l+m+mi}{1}\PY{p}{,} \PY{l+m+mi}{2}\PY{p}{)}
\PY{n}{plt}\PY{o}{.}\PY{n}{plot}\PY{p}{(}\PY{n}{t\PYZus{}avg}\PY{p}{,} \PY{n}{v\PYZus{}avg}\PY{p}{,} \PY{l+s+s1}{\PYZsq{}}\PY{l+s+s1}{b\PYZhy{}}\PY{l+s+s1}{\PYZsq{}}\PY{p}{,} \PY{n}{label}\PY{o}{=}\PY{l+s+s1}{\PYZsq{}}\PY{l+s+s1}{平均加速度法 (β=1/4)}\PY{l+s+s1}{\PYZsq{}}\PY{p}{)}
\PY{n}{plt}\PY{o}{.}\PY{n}{plot}\PY{p}{(}\PY{n}{t\PYZus{}lin}\PY{p}{,} \PY{n}{v\PYZus{}lin}\PY{p}{,} \PY{l+s+s1}{\PYZsq{}}\PY{l+s+s1}{r\PYZhy{}\PYZhy{}}\PY{l+s+s1}{\PYZsq{}}\PY{p}{,} \PY{n}{label}\PY{o}{=}\PY{l+s+s1}{\PYZsq{}}\PY{l+s+s1}{线性加速度法 (β=1/6)}\PY{l+s+s1}{\PYZsq{}}\PY{p}{)}
\PY{n}{plt}\PY{o}{.}\PY{n}{xlabel}\PY{p}{(}\PY{l+s+s1}{\PYZsq{}}\PY{l+s+s1}{时间 (秒)}\PY{l+s+s1}{\PYZsq{}}\PY{p}{)}
\PY{n}{plt}\PY{o}{.}\PY{n}{ylabel}\PY{p}{(}\PY{l+s+s1}{\PYZsq{}}\PY{l+s+s1}{速度 (m/s)}\PY{l+s+s1}{\PYZsq{}}\PY{p}{)}
\PY{n}{plt}\PY{o}{.}\PY{n}{title}\PY{p}{(}\PY{l+s+sa}{f}\PY{l+s+s1}{\PYZsq{}}\PY{l+s+s1}{不同Newmark参数下的速度响应对比 (dt = }\PY{l+s+si}{\PYZob{}}\PY{n}{dt\PYZus{}large}\PY{l+s+si}{:}\PY{l+s+s1}{.4f}\PY{l+s+si}{\PYZcb{}}\PY{l+s+s1}{秒)}\PY{l+s+s1}{\PYZsq{}}\PY{p}{)}
\PY{n}{plt}\PY{o}{.}\PY{n}{legend}\PY{p}{(}\PY{p}{)}
\PY{n}{plt}\PY{o}{.}\PY{n}{grid}\PY{p}{(}\PY{k+kc}{True}\PY{p}{)}

\PY{n}{plt}\PY{o}{.}\PY{n}{tight\PYZus{}layout}\PY{p}{(}\PY{p}{)}
\PY{n}{plt}\PY{o}{.}\PY{n}{show}\PY{p}{(}\PY{p}{)}
\end{Verbatim}
\end{tcolorbox}

    \begin{Verbatim}[commandchars=\\\{\}]
/var/folders/tm/snrp8gcn00113mpkf\_j5jmqm0000gn/T/ipykernel\_8490/3158952429.py:35
: UserWarning: Glyph 26102 (\textbackslash{}N\{CJK UNIFIED IDEOGRAPH-65F6\}) missing from current
font.
  plt.tight\_layout()
/var/folders/tm/snrp8gcn00113mpkf\_j5jmqm0000gn/T/ipykernel\_8490/3158952429.py:35
: UserWarning: Glyph 38388 (\textbackslash{}N\{CJK UNIFIED IDEOGRAPH-95F4\}) missing from current
font.
  plt.tight\_layout()
/var/folders/tm/snrp8gcn00113mpkf\_j5jmqm0000gn/T/ipykernel\_8490/3158952429.py:35
: UserWarning: Glyph 31186 (\textbackslash{}N\{CJK UNIFIED IDEOGRAPH-79D2\}) missing from current
font.
  plt.tight\_layout()
/var/folders/tm/snrp8gcn00113mpkf\_j5jmqm0000gn/T/ipykernel\_8490/3158952429.py:35
: UserWarning: Glyph 20301 (\textbackslash{}N\{CJK UNIFIED IDEOGRAPH-4F4D\}) missing from current
font.
  plt.tight\_layout()
/var/folders/tm/snrp8gcn00113mpkf\_j5jmqm0000gn/T/ipykernel\_8490/3158952429.py:35
: UserWarning: Glyph 31227 (\textbackslash{}N\{CJK UNIFIED IDEOGRAPH-79FB\}) missing from current
font.
  plt.tight\_layout()
/var/folders/tm/snrp8gcn00113mpkf\_j5jmqm0000gn/T/ipykernel\_8490/3158952429.py:35
: UserWarning: Glyph 19981 (\textbackslash{}N\{CJK UNIFIED IDEOGRAPH-4E0D\}) missing from current
font.
  plt.tight\_layout()
/var/folders/tm/snrp8gcn00113mpkf\_j5jmqm0000gn/T/ipykernel\_8490/3158952429.py:35
: UserWarning: Glyph 21516 (\textbackslash{}N\{CJK UNIFIED IDEOGRAPH-540C\}) missing from current
font.
  plt.tight\_layout()
/var/folders/tm/snrp8gcn00113mpkf\_j5jmqm0000gn/T/ipykernel\_8490/3158952429.py:35
: UserWarning: Glyph 21442 (\textbackslash{}N\{CJK UNIFIED IDEOGRAPH-53C2\}) missing from current
font.
  plt.tight\_layout()
/var/folders/tm/snrp8gcn00113mpkf\_j5jmqm0000gn/T/ipykernel\_8490/3158952429.py:35
: UserWarning: Glyph 25968 (\textbackslash{}N\{CJK UNIFIED IDEOGRAPH-6570\}) missing from current
font.
  plt.tight\_layout()
/var/folders/tm/snrp8gcn00113mpkf\_j5jmqm0000gn/T/ipykernel\_8490/3158952429.py:35
: UserWarning: Glyph 19979 (\textbackslash{}N\{CJK UNIFIED IDEOGRAPH-4E0B\}) missing from current
font.
  plt.tight\_layout()
/var/folders/tm/snrp8gcn00113mpkf\_j5jmqm0000gn/T/ipykernel\_8490/3158952429.py:35
: UserWarning: Glyph 30340 (\textbackslash{}N\{CJK UNIFIED IDEOGRAPH-7684\}) missing from current
font.
  plt.tight\_layout()
/var/folders/tm/snrp8gcn00113mpkf\_j5jmqm0000gn/T/ipykernel\_8490/3158952429.py:35
: UserWarning: Glyph 21709 (\textbackslash{}N\{CJK UNIFIED IDEOGRAPH-54CD\}) missing from current
font.
  plt.tight\_layout()
/var/folders/tm/snrp8gcn00113mpkf\_j5jmqm0000gn/T/ipykernel\_8490/3158952429.py:35
: UserWarning: Glyph 24212 (\textbackslash{}N\{CJK UNIFIED IDEOGRAPH-5E94\}) missing from current
font.
  plt.tight\_layout()
/var/folders/tm/snrp8gcn00113mpkf\_j5jmqm0000gn/T/ipykernel\_8490/3158952429.py:35
: UserWarning: Glyph 23545 (\textbackslash{}N\{CJK UNIFIED IDEOGRAPH-5BF9\}) missing from current
font.
  plt.tight\_layout()
/var/folders/tm/snrp8gcn00113mpkf\_j5jmqm0000gn/T/ipykernel\_8490/3158952429.py:35
: UserWarning: Glyph 27604 (\textbackslash{}N\{CJK UNIFIED IDEOGRAPH-6BD4\}) missing from current
font.
  plt.tight\_layout()
/var/folders/tm/snrp8gcn00113mpkf\_j5jmqm0000gn/T/ipykernel\_8490/3158952429.py:35
: UserWarning: Glyph 24179 (\textbackslash{}N\{CJK UNIFIED IDEOGRAPH-5E73\}) missing from current
font.
  plt.tight\_layout()
/var/folders/tm/snrp8gcn00113mpkf\_j5jmqm0000gn/T/ipykernel\_8490/3158952429.py:35
: UserWarning: Glyph 22343 (\textbackslash{}N\{CJK UNIFIED IDEOGRAPH-5747\}) missing from current
font.
  plt.tight\_layout()
/var/folders/tm/snrp8gcn00113mpkf\_j5jmqm0000gn/T/ipykernel\_8490/3158952429.py:35
: UserWarning: Glyph 21152 (\textbackslash{}N\{CJK UNIFIED IDEOGRAPH-52A0\}) missing from current
font.
  plt.tight\_layout()
/var/folders/tm/snrp8gcn00113mpkf\_j5jmqm0000gn/T/ipykernel\_8490/3158952429.py:35
: UserWarning: Glyph 36895 (\textbackslash{}N\{CJK UNIFIED IDEOGRAPH-901F\}) missing from current
font.
  plt.tight\_layout()
/var/folders/tm/snrp8gcn00113mpkf\_j5jmqm0000gn/T/ipykernel\_8490/3158952429.py:35
: UserWarning: Glyph 24230 (\textbackslash{}N\{CJK UNIFIED IDEOGRAPH-5EA6\}) missing from current
font.
  plt.tight\_layout()
/var/folders/tm/snrp8gcn00113mpkf\_j5jmqm0000gn/T/ipykernel\_8490/3158952429.py:35
: UserWarning: Glyph 27861 (\textbackslash{}N\{CJK UNIFIED IDEOGRAPH-6CD5\}) missing from current
font.
  plt.tight\_layout()
/var/folders/tm/snrp8gcn00113mpkf\_j5jmqm0000gn/T/ipykernel\_8490/3158952429.py:35
: UserWarning: Glyph 32447 (\textbackslash{}N\{CJK UNIFIED IDEOGRAPH-7EBF\}) missing from current
font.
  plt.tight\_layout()
/var/folders/tm/snrp8gcn00113mpkf\_j5jmqm0000gn/T/ipykernel\_8490/3158952429.py:35
: UserWarning: Glyph 24615 (\textbackslash{}N\{CJK UNIFIED IDEOGRAPH-6027\}) missing from current
font.
  plt.tight\_layout()
/Users/henri/miniconda3/envs/henri\_env/lib/python3.10/site-
packages/IPython/core/pylabtools.py:152: UserWarning: Glyph 21442 (\textbackslash{}N\{CJK
UNIFIED IDEOGRAPH-53C2\}) missing from current font.
  fig.canvas.print\_figure(bytes\_io, **kw)
/Users/henri/miniconda3/envs/henri\_env/lib/python3.10/site-
packages/IPython/core/pylabtools.py:152: UserWarning: Glyph 25968 (\textbackslash{}N\{CJK
UNIFIED IDEOGRAPH-6570\}) missing from current font.
  fig.canvas.print\_figure(bytes\_io, **kw)
/Users/henri/miniconda3/envs/henri\_env/lib/python3.10/site-
packages/IPython/core/pylabtools.py:152: UserWarning: Glyph 32447 (\textbackslash{}N\{CJK
UNIFIED IDEOGRAPH-7EBF\}) missing from current font.
  fig.canvas.print\_figure(bytes\_io, **kw)
/Users/henri/miniconda3/envs/henri\_env/lib/python3.10/site-
packages/IPython/core/pylabtools.py:152: UserWarning: Glyph 24615 (\textbackslash{}N\{CJK
UNIFIED IDEOGRAPH-6027\}) missing from current font.
  fig.canvas.print\_figure(bytes\_io, **kw)
    \end{Verbatim}

    \begin{center}
    \adjustimage{max size={0.9\linewidth}{0.9\paperheight}}{output_12_1.png}
    \end{center}
    { \hspace*{\fill} \\}
    
    \subsection{7.
与精确解进行对比}\label{ux4e0eux7cbeux786eux89e3ux8fdbux884cux5bf9ux6bd4}

为了验证Newmark方法的准确性,我们将其结果与scipy.integrate.solve\_ivp的高精度解法进行对比:

    \begin{tcolorbox}[breakable, size=fbox, boxrule=1pt, pad at break*=1mm,colback=cellbackground, colframe=cellborder]
\prompt{In}{incolor}{7}{\boxspacing}
\begin{Verbatim}[commandchars=\\\{\}]
\PY{c+c1}{\PYZsh{} 定义微分方程组 (一阶方程组形式)}
\PY{k}{def} \PY{n+nf}{sdof\PYZus{}system}\PY{p}{(}\PY{n}{t}\PY{p}{,} \PY{n}{y}\PY{p}{)}\PY{p}{:}
    \PY{n}{u}\PY{p}{,} \PY{n}{v} \PY{o}{=} \PY{n}{y}  \PY{c+c1}{\PYZsh{} y = [位移, 速度]}
    \PY{n}{dvdt} \PY{o}{=} \PY{p}{(}\PY{n}{force\PYZus{}function}\PY{p}{(}\PY{n}{t}\PY{p}{)} \PY{o}{\PYZhy{}} \PY{n}{c} \PY{o}{*} \PY{n}{v} \PY{o}{\PYZhy{}} \PY{n}{k} \PY{o}{*} \PY{n}{u}\PY{p}{)} \PY{o}{/} \PY{n}{m}  \PY{c+c1}{\PYZsh{} 加速度}
    \PY{k}{return} \PY{p}{[}\PY{n}{v}\PY{p}{,} \PY{n}{dvdt}\PY{p}{]}

\PY{c+c1}{\PYZsh{} 初始条件}
\PY{n}{y0} \PY{o}{=} \PY{p}{[}\PY{n}{u0}\PY{p}{,} \PY{n}{v0}\PY{p}{]}

\PY{c+c1}{\PYZsh{} 使用scipy的高精度解法求解}
\PY{n}{sol} \PY{o}{=} \PY{n}{solve\PYZus{}ivp}\PY{p}{(}\PY{n}{sdof\PYZus{}system}\PY{p}{,} \PY{p}{[}\PY{l+m+mi}{0}\PY{p}{,} \PY{n}{t\PYZus{}final}\PY{p}{]}\PY{p}{,} \PY{n}{y0}\PY{p}{,} \PY{n}{method}\PY{o}{=}\PY{l+s+s1}{\PYZsq{}}\PY{l+s+s1}{RK45}\PY{l+s+s1}{\PYZsq{}}\PY{p}{,} \PY{n}{rtol}\PY{o}{=}\PY{l+m+mf}{1e\PYZhy{}8}\PY{p}{,} \PY{n}{atol}\PY{o}{=}\PY{l+m+mf}{1e\PYZhy{}8}\PY{p}{,} \PY{n}{dense\PYZus{}output}\PY{o}{=}\PY{k+kc}{True}\PY{p}{)}

\PY{c+c1}{\PYZsh{} 在与Newmark法相同的时间点上评估解}
\PY{n}{t\PYZus{}fine} \PY{o}{=} \PY{n}{np}\PY{o}{.}\PY{n}{linspace}\PY{p}{(}\PY{l+m+mi}{0}\PY{p}{,} \PY{n}{t\PYZus{}final}\PY{p}{,} \PY{l+m+mi}{1000}\PY{p}{)}
\PY{n}{y\PYZus{}fine} \PY{o}{=} \PY{n}{sol}\PY{o}{.}\PY{n}{sol}\PY{p}{(}\PY{n}{t\PYZus{}fine}\PY{p}{)}
\PY{n}{u\PYZus{}exact} \PY{o}{=} \PY{n}{y\PYZus{}fine}\PY{p}{[}\PY{l+m+mi}{0}\PY{p}{]}
\PY{n}{v\PYZus{}exact} \PY{o}{=} \PY{n}{y\PYZus{}fine}\PY{p}{[}\PY{l+m+mi}{1}\PY{p}{]}

\PY{c+c1}{\PYZsh{} 使用较小步长的Newmark法结果作为对比}
\PY{n}{dt\PYZus{}small} \PY{o}{=} \PY{n}{T\PYZus{}n} \PY{o}{/} \PY{l+m+mi}{100}
\PY{n}{t\PYZus{}nm}\PY{p}{,} \PY{n}{u\PYZus{}nm}\PY{p}{,} \PY{n}{v\PYZus{}nm}\PY{p}{,} \PY{n}{a\PYZus{}nm} \PY{o}{=} \PY{n}{newmark\PYZus{}method}\PY{p}{(}\PY{n}{m}\PY{p}{,} \PY{n}{c}\PY{p}{,} \PY{n}{k}\PY{p}{,} \PY{n}{force\PYZus{}function}\PY{p}{,} \PY{n}{u0}\PY{p}{,} \PY{n}{v0}\PY{p}{,} \PY{n}{t\PYZus{}final}\PY{p}{,} \PY{n}{dt\PYZus{}small}\PY{p}{,} \PY{n}{beta\PYZus{}avg}\PY{p}{,} \PY{n}{gamma\PYZus{}avg}\PY{p}{)}

\PY{c+c1}{\PYZsh{} 对比可视化}
\PY{n}{plt}\PY{o}{.}\PY{n}{figure}\PY{p}{(}\PY{n}{figsize}\PY{o}{=}\PY{p}{(}\PY{l+m+mi}{14}\PY{p}{,} \PY{l+m+mi}{10}\PY{p}{)}\PY{p}{)}

\PY{n}{plt}\PY{o}{.}\PY{n}{subplot}\PY{p}{(}\PY{l+m+mi}{2}\PY{p}{,} \PY{l+m+mi}{1}\PY{p}{,} \PY{l+m+mi}{1}\PY{p}{)}
\PY{n}{plt}\PY{o}{.}\PY{n}{plot}\PY{p}{(}\PY{n}{t\PYZus{}fine}\PY{p}{,} \PY{n}{u\PYZus{}exact}\PY{p}{,} \PY{l+s+s1}{\PYZsq{}}\PY{l+s+s1}{k\PYZhy{}}\PY{l+s+s1}{\PYZsq{}}\PY{p}{,} \PY{n}{label}\PY{o}{=}\PY{l+s+s1}{\PYZsq{}}\PY{l+s+s1}{精确解 (RK45)}\PY{l+s+s1}{\PYZsq{}}\PY{p}{)}
\PY{n}{plt}\PY{o}{.}\PY{n}{plot}\PY{p}{(}\PY{n}{t\PYZus{}nm}\PY{p}{,} \PY{n}{u\PYZus{}nm}\PY{p}{,} \PY{l+s+s1}{\PYZsq{}}\PY{l+s+s1}{r\PYZhy{}\PYZhy{}}\PY{l+s+s1}{\PYZsq{}}\PY{p}{,} \PY{n}{label}\PY{o}{=}\PY{l+s+s1}{\PYZsq{}}\PY{l+s+s1}{Newmark法 (平均加速度)}\PY{l+s+s1}{\PYZsq{}}\PY{p}{)}
\PY{n}{plt}\PY{o}{.}\PY{n}{xlabel}\PY{p}{(}\PY{l+s+s1}{\PYZsq{}}\PY{l+s+s1}{时间 (秒)}\PY{l+s+s1}{\PYZsq{}}\PY{p}{)}
\PY{n}{plt}\PY{o}{.}\PY{n}{ylabel}\PY{p}{(}\PY{l+s+s1}{\PYZsq{}}\PY{l+s+s1}{位移 (m)}\PY{l+s+s1}{\PYZsq{}}\PY{p}{)}
\PY{n}{plt}\PY{o}{.}\PY{n}{title}\PY{p}{(}\PY{l+s+s1}{\PYZsq{}}\PY{l+s+s1}{Newmark法与精确解的位移对比}\PY{l+s+s1}{\PYZsq{}}\PY{p}{)}
\PY{n}{plt}\PY{o}{.}\PY{n}{legend}\PY{p}{(}\PY{p}{)}
\PY{n}{plt}\PY{o}{.}\PY{n}{grid}\PY{p}{(}\PY{k+kc}{True}\PY{p}{)}

\PY{n}{plt}\PY{o}{.}\PY{n}{subplot}\PY{p}{(}\PY{l+m+mi}{2}\PY{p}{,} \PY{l+m+mi}{1}\PY{p}{,} \PY{l+m+mi}{2}\PY{p}{)}
\PY{n}{plt}\PY{o}{.}\PY{n}{plot}\PY{p}{(}\PY{n}{t\PYZus{}fine}\PY{p}{,} \PY{n}{v\PYZus{}exact}\PY{p}{,} \PY{l+s+s1}{\PYZsq{}}\PY{l+s+s1}{k\PYZhy{}}\PY{l+s+s1}{\PYZsq{}}\PY{p}{,} \PY{n}{label}\PY{o}{=}\PY{l+s+s1}{\PYZsq{}}\PY{l+s+s1}{精确解 (RK45)}\PY{l+s+s1}{\PYZsq{}}\PY{p}{)}
\PY{n}{plt}\PY{o}{.}\PY{n}{plot}\PY{p}{(}\PY{n}{t\PYZus{}nm}\PY{p}{,} \PY{n}{v\PYZus{}nm}\PY{p}{,} \PY{l+s+s1}{\PYZsq{}}\PY{l+s+s1}{r\PYZhy{}\PYZhy{}}\PY{l+s+s1}{\PYZsq{}}\PY{p}{,} \PY{n}{label}\PY{o}{=}\PY{l+s+s1}{\PYZsq{}}\PY{l+s+s1}{Newmark法 (平均加速度)}\PY{l+s+s1}{\PYZsq{}}\PY{p}{)}
\PY{n}{plt}\PY{o}{.}\PY{n}{xlabel}\PY{p}{(}\PY{l+s+s1}{\PYZsq{}}\PY{l+s+s1}{时间 (秒)}\PY{l+s+s1}{\PYZsq{}}\PY{p}{)}
\PY{n}{plt}\PY{o}{.}\PY{n}{ylabel}\PY{p}{(}\PY{l+s+s1}{\PYZsq{}}\PY{l+s+s1}{速度 (m/s)}\PY{l+s+s1}{\PYZsq{}}\PY{p}{)}
\PY{n}{plt}\PY{o}{.}\PY{n}{title}\PY{p}{(}\PY{l+s+s1}{\PYZsq{}}\PY{l+s+s1}{Newmark法与精确解的速度对比}\PY{l+s+s1}{\PYZsq{}}\PY{p}{)}
\PY{n}{plt}\PY{o}{.}\PY{n}{legend}\PY{p}{(}\PY{p}{)}
\PY{n}{plt}\PY{o}{.}\PY{n}{grid}\PY{p}{(}\PY{k+kc}{True}\PY{p}{)}

\PY{n}{plt}\PY{o}{.}\PY{n}{tight\PYZus{}layout}\PY{p}{(}\PY{p}{)}
\PY{n}{plt}\PY{o}{.}\PY{n}{show}\PY{p}{(}\PY{p}{)}

\PY{c+c1}{\PYZsh{} 计算误差}
\PY{n}{u\PYZus{}nm\PYZus{}interp} \PY{o}{=} \PY{n}{np}\PY{o}{.}\PY{n}{interp}\PY{p}{(}\PY{n}{t\PYZus{}fine}\PY{p}{,} \PY{n}{t\PYZus{}nm}\PY{p}{,} \PY{n}{u\PYZus{}nm}\PY{p}{)}
\PY{n}{v\PYZus{}nm\PYZus{}interp} \PY{o}{=} \PY{n}{np}\PY{o}{.}\PY{n}{interp}\PY{p}{(}\PY{n}{t\PYZus{}fine}\PY{p}{,} \PY{n}{t\PYZus{}nm}\PY{p}{,} \PY{n}{v\PYZus{}nm}\PY{p}{)}

\PY{n}{u\PYZus{}error} \PY{o}{=} \PY{n}{np}\PY{o}{.}\PY{n}{abs}\PY{p}{(}\PY{n}{u\PYZus{}nm\PYZus{}interp} \PY{o}{\PYZhy{}} \PY{n}{u\PYZus{}exact}\PY{p}{)}
\PY{n}{v\PYZus{}error} \PY{o}{=} \PY{n}{np}\PY{o}{.}\PY{n}{abs}\PY{p}{(}\PY{n}{v\PYZus{}nm\PYZus{}interp} \PY{o}{\PYZhy{}} \PY{n}{v\PYZus{}exact}\PY{p}{)}

\PY{n+nb}{print}\PY{p}{(}\PY{l+s+sa}{f}\PY{l+s+s2}{\PYZdq{}}\PY{l+s+s2}{位移最大误差: }\PY{l+s+si}{\PYZob{}}\PY{n}{np}\PY{o}{.}\PY{n}{max}\PY{p}{(}\PY{n}{u\PYZus{}error}\PY{p}{)}\PY{l+s+si}{:}\PY{l+s+s2}{.8f}\PY{l+s+si}{\PYZcb{}}\PY{l+s+s2}{ m}\PY{l+s+s2}{\PYZdq{}}\PY{p}{)}
\PY{n+nb}{print}\PY{p}{(}\PY{l+s+sa}{f}\PY{l+s+s2}{\PYZdq{}}\PY{l+s+s2}{速度最大误差: }\PY{l+s+si}{\PYZob{}}\PY{n}{np}\PY{o}{.}\PY{n}{max}\PY{p}{(}\PY{n}{v\PYZus{}error}\PY{p}{)}\PY{l+s+si}{:}\PY{l+s+s2}{.8f}\PY{l+s+si}{\PYZcb{}}\PY{l+s+s2}{ m/s}\PY{l+s+s2}{\PYZdq{}}\PY{p}{)}
\PY{n+nb}{print}\PY{p}{(}\PY{l+s+sa}{f}\PY{l+s+s2}{\PYZdq{}}\PY{l+s+s2}{相对位移误差: }\PY{l+s+si}{\PYZob{}}\PY{l+m+mi}{100}\PY{+w}{ }\PY{o}{*}\PY{+w}{ }\PY{n}{np}\PY{o}{.}\PY{n}{max}\PY{p}{(}\PY{n}{u\PYZus{}error}\PY{p}{)}\PY{+w}{ }\PY{o}{/}\PY{+w}{ }\PY{n}{np}\PY{o}{.}\PY{n}{max}\PY{p}{(}\PY{n}{np}\PY{o}{.}\PY{n}{abs}\PY{p}{(}\PY{n}{u\PYZus{}exact}\PY{p}{)}\PY{p}{)}\PY{l+s+si}{:}\PY{l+s+s2}{.6f}\PY{l+s+si}{\PYZcb{}}\PY{l+s+s2}{\PYZpc{}}\PY{l+s+s2}{\PYZdq{}}\PY{p}{)}
\end{Verbatim}
\end{tcolorbox}

    \begin{Verbatim}[commandchars=\\\{\}]
/var/folders/tm/snrp8gcn00113mpkf\_j5jmqm0000gn/T/ipykernel\_8490/1135938699.py:44
: UserWarning: Glyph 26102 (\textbackslash{}N\{CJK UNIFIED IDEOGRAPH-65F6\}) missing from current
font.
  plt.tight\_layout()
/var/folders/tm/snrp8gcn00113mpkf\_j5jmqm0000gn/T/ipykernel\_8490/1135938699.py:44
: UserWarning: Glyph 38388 (\textbackslash{}N\{CJK UNIFIED IDEOGRAPH-95F4\}) missing from current
font.
  plt.tight\_layout()
/var/folders/tm/snrp8gcn00113mpkf\_j5jmqm0000gn/T/ipykernel\_8490/1135938699.py:44
: UserWarning: Glyph 31186 (\textbackslash{}N\{CJK UNIFIED IDEOGRAPH-79D2\}) missing from current
font.
  plt.tight\_layout()
/var/folders/tm/snrp8gcn00113mpkf\_j5jmqm0000gn/T/ipykernel\_8490/1135938699.py:44
: UserWarning: Glyph 20301 (\textbackslash{}N\{CJK UNIFIED IDEOGRAPH-4F4D\}) missing from current
font.
  plt.tight\_layout()
/var/folders/tm/snrp8gcn00113mpkf\_j5jmqm0000gn/T/ipykernel\_8490/1135938699.py:44
: UserWarning: Glyph 31227 (\textbackslash{}N\{CJK UNIFIED IDEOGRAPH-79FB\}) missing from current
font.
  plt.tight\_layout()
/var/folders/tm/snrp8gcn00113mpkf\_j5jmqm0000gn/T/ipykernel\_8490/1135938699.py:44
: UserWarning: Glyph 27861 (\textbackslash{}N\{CJK UNIFIED IDEOGRAPH-6CD5\}) missing from current
font.
  plt.tight\_layout()
/var/folders/tm/snrp8gcn00113mpkf\_j5jmqm0000gn/T/ipykernel\_8490/1135938699.py:44
: UserWarning: Glyph 19982 (\textbackslash{}N\{CJK UNIFIED IDEOGRAPH-4E0E\}) missing from current
font.
  plt.tight\_layout()
/var/folders/tm/snrp8gcn00113mpkf\_j5jmqm0000gn/T/ipykernel\_8490/1135938699.py:44
: UserWarning: Glyph 31934 (\textbackslash{}N\{CJK UNIFIED IDEOGRAPH-7CBE\}) missing from current
font.
  plt.tight\_layout()
/var/folders/tm/snrp8gcn00113mpkf\_j5jmqm0000gn/T/ipykernel\_8490/1135938699.py:44
: UserWarning: Glyph 30830 (\textbackslash{}N\{CJK UNIFIED IDEOGRAPH-786E\}) missing from current
font.
  plt.tight\_layout()
/var/folders/tm/snrp8gcn00113mpkf\_j5jmqm0000gn/T/ipykernel\_8490/1135938699.py:44
: UserWarning: Glyph 35299 (\textbackslash{}N\{CJK UNIFIED IDEOGRAPH-89E3\}) missing from current
font.
  plt.tight\_layout()
/var/folders/tm/snrp8gcn00113mpkf\_j5jmqm0000gn/T/ipykernel\_8490/1135938699.py:44
: UserWarning: Glyph 30340 (\textbackslash{}N\{CJK UNIFIED IDEOGRAPH-7684\}) missing from current
font.
  plt.tight\_layout()
/var/folders/tm/snrp8gcn00113mpkf\_j5jmqm0000gn/T/ipykernel\_8490/1135938699.py:44
: UserWarning: Glyph 23545 (\textbackslash{}N\{CJK UNIFIED IDEOGRAPH-5BF9\}) missing from current
font.
  plt.tight\_layout()
/var/folders/tm/snrp8gcn00113mpkf\_j5jmqm0000gn/T/ipykernel\_8490/1135938699.py:44
: UserWarning: Glyph 27604 (\textbackslash{}N\{CJK UNIFIED IDEOGRAPH-6BD4\}) missing from current
font.
  plt.tight\_layout()
/var/folders/tm/snrp8gcn00113mpkf\_j5jmqm0000gn/T/ipykernel\_8490/1135938699.py:44
: UserWarning: Glyph 24179 (\textbackslash{}N\{CJK UNIFIED IDEOGRAPH-5E73\}) missing from current
font.
  plt.tight\_layout()
/var/folders/tm/snrp8gcn00113mpkf\_j5jmqm0000gn/T/ipykernel\_8490/1135938699.py:44
: UserWarning: Glyph 22343 (\textbackslash{}N\{CJK UNIFIED IDEOGRAPH-5747\}) missing from current
font.
  plt.tight\_layout()
/var/folders/tm/snrp8gcn00113mpkf\_j5jmqm0000gn/T/ipykernel\_8490/1135938699.py:44
: UserWarning: Glyph 21152 (\textbackslash{}N\{CJK UNIFIED IDEOGRAPH-52A0\}) missing from current
font.
  plt.tight\_layout()
/var/folders/tm/snrp8gcn00113mpkf\_j5jmqm0000gn/T/ipykernel\_8490/1135938699.py:44
: UserWarning: Glyph 36895 (\textbackslash{}N\{CJK UNIFIED IDEOGRAPH-901F\}) missing from current
font.
  plt.tight\_layout()
/var/folders/tm/snrp8gcn00113mpkf\_j5jmqm0000gn/T/ipykernel\_8490/1135938699.py:44
: UserWarning: Glyph 24230 (\textbackslash{}N\{CJK UNIFIED IDEOGRAPH-5EA6\}) missing from current
font.
  plt.tight\_layout()
/Users/henri/miniconda3/envs/henri\_env/lib/python3.10/site-
packages/IPython/core/pylabtools.py:152: UserWarning: Glyph 19982 (\textbackslash{}N\{CJK
UNIFIED IDEOGRAPH-4E0E\}) missing from current font.
  fig.canvas.print\_figure(bytes\_io, **kw)
/Users/henri/miniconda3/envs/henri\_env/lib/python3.10/site-
packages/IPython/core/pylabtools.py:152: UserWarning: Glyph 31934 (\textbackslash{}N\{CJK
UNIFIED IDEOGRAPH-7CBE\}) missing from current font.
  fig.canvas.print\_figure(bytes\_io, **kw)
/Users/henri/miniconda3/envs/henri\_env/lib/python3.10/site-
packages/IPython/core/pylabtools.py:152: UserWarning: Glyph 30830 (\textbackslash{}N\{CJK
UNIFIED IDEOGRAPH-786E\}) missing from current font.
  fig.canvas.print\_figure(bytes\_io, **kw)
/Users/henri/miniconda3/envs/henri\_env/lib/python3.10/site-
packages/IPython/core/pylabtools.py:152: UserWarning: Glyph 35299 (\textbackslash{}N\{CJK
UNIFIED IDEOGRAPH-89E3\}) missing from current font.
  fig.canvas.print\_figure(bytes\_io, **kw)
    \end{Verbatim}

    \begin{center}
    \adjustimage{max size={0.9\linewidth}{0.9\paperheight}}{output_14_1.png}
    \end{center}
    { \hspace*{\fill} \\}
    
    \begin{Verbatim}[commandchars=\\\{\}]
位移最大误差: 0.17106874 m
速度最大误差: 1.53166000 m/s
相对位移误差: 45.917078\%
    \end{Verbatim}

    \subsection{8.
Newmark方法的特点总结}\label{newmarkux65b9ux6cd5ux7684ux7279ux70b9ux603bux7ed3}

通过以上算例展示,我们可以总结Newmark方法的以下特点:

\begin{enumerate}
\def\labelenumi{\arabic{enumi}.}
\tightlist
\item
  \textbf{稳定性}:

  \begin{itemize}
  \tightlist
  \item
    当 β ≥ γ/2 ≥ 1/4 时,方法是无条件稳定的
  \item
    平均加速度法(β=1/4, γ=1/2)无条件稳定
  \item
    线性加速度法(β=1/6, γ=1/2)是条件稳定的,需要满足时间步长限制
  \end{itemize}
\item
  \textbf{精度}:

  \begin{itemize}
  \tightlist
  \item
    平均加速度法是二阶精度的
  \item
    时间步长越小,数值解越接近精确解
  \end{itemize}
\item
  \textbf{计算效率}:

  \begin{itemize}
  \tightlist
  \item
    作为隐式方法,每一步需要求解方程组
  \item
    对于线性系统,可以预先分解系统矩阵提高效率
  \item
    对于非线性系统,需要在每个时间步内迭代求解
  \end{itemize}
\item
  \textbf{参数选择}:

  \begin{itemize}
  \tightlist
  \item
    β和γ参数的选择影响方法的稳定性、精度和数值阻尼
  \item
    常用组合:

    \begin{itemize}
    \tightlist
    \item
      β=1/4, γ=1/2:平均加速度法,无条件稳定,无数值阻尼
    \item
      β=1/6, γ=1/2:线性加速度法,条件稳定,无数值阻尼
    \item
      β=(2γ-1)/4, γ\textgreater1/2:引入数值阻尼的方法
    \end{itemize}
  \end{itemize}
\item
  \textbf{实际应用建议}:

  \begin{itemize}
  \tightlist
  \item
    对于大多数结构动力学问题,平均加速度法是最常用的,因为其无条件稳定
  \item
    时间步长通常选择为结构最高频率周期的1/10\textasciitilde1/20
  \item
    对于需要控制高频响应的问题,可以适当引入数值阻尼
  \end{itemize}
\end{enumerate}

    \subsection{9. 结论}\label{ux7ed3ux8bba}

Newmark方法作为一种经典的结构动力学数值积分方法,具有实现简单、适用性广的特点。通过合理选择参数β和γ,可以在稳定性和精度之间取得平衡。在实际工程中,平均加速度法(β=1/4,
γ=1/2)因其无条件稳定的特性被广泛采用。

本算例通过单自由度系统的动力响应分析,直观展示了Newmark方法的实现过程及其特性,特别是不同时间步长和参数选择对计算结果的影响。这些原理同样适用于多自由度系统分析,只需将标量方程扩展为矩阵形式。


    % Add a bibliography block to the postdoc
    
    
    
\end{document}
