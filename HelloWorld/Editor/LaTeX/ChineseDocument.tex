% !TeX program = xelatex
% !TeX encoding = UTF-8

\documentclass[12pt, a4paper]{ctexart} % 使用ctexart类支持中文

% 引入必要的包
\usepackage{amsmath,amssymb,amsfonts} % 数学公式支持
\usepackage{graphicx} % 图片支持
\usepackage{booktabs} % 美观的表格
\usepackage{hyperref} % 超链接支持
\usepackage{geometry} % 页面设置
\usepackage{xcolor} % 颜色支持
\usepackage{listings} % 代码列表
\usepackage{caption} % 图表标题
\usepackage{subcaption} % 子图表支持
\usepackage{siunitx} % 国际单位制
\usepackage{biblatex} % 文献引用支持
\addbibresource{references.bib} % 引入参考文献文件

% 页面设置
\geometry{
  top=2.5cm,
  bottom=2.5cm,
  left=2.5cm,
  right=2.5cm
}

% 超链接设置
\hypersetup{
  colorlinks=true,
  linkcolor=blue,
  filecolor=magenta,
  urlcolor=cyan,
}

% 文档信息
\title{中文\LaTeX{}文档示例}
\author{作者名称}
\date{\today}

\begin{document}

\maketitle % 显示标题

\begin{abstract}
这是一个支持中文的\LaTeX{}文档示例,展示了如何插入数学公式、表格、图片以及参考文献的引用方法。本文档使用XeLaTeX编译以获得最佳的中文支持效果。
\end{abstract}

\tableofcontents % 生成目录
\newpage

\section{引言}
\LaTeX{}是一种基于\TeX{}的排版系统,常用于生成高质量的科技论文、书籍和其他文档。中文\LaTeX{}文档可以通过使用ctex宏包或ctexart文档类来实现。

\section{数学公式示例}
\LaTeX{}最大的优势之一是对数学公式的优秀支持。以下展示了一些数学公式的例子:

\subsection{行内公式}
爱因斯坦的质能方程 $E=mc^2$ 表明质量和能量是等价的。欧拉公式 $e^{i\pi}+1=0$ 被认为是数学中最美的公式之一。

\subsection{行间公式}
牛顿第二定律可以表示为:
\begin{equation}
  \vec{F} = m\vec{a}
\end{equation}

拉格朗日力学中的欧拉-拉格朗日方程:
\begin{equation}
  \frac{d}{dt}\left(\frac{\partial L}{\partial \dot{q}_i}\right) - \frac{\partial L}{\partial q_i} = 0
\end{equation}

多行公式对齐示例:
\begin{align}
  (a+b)^2 &= (a+b)(a+b) \\
  &= a^2 + ab + ba + b^2 \\
  &= a^2 + 2ab + b^2
\end{align}

积分表达式:
\begin{equation}
  \int_{0}^{\infty} e^{-x^2} dx = \frac{\sqrt{\pi}}{2}
\end{equation}

矩阵示例:
\begin{equation}
  A = \begin{pmatrix}
    a_{11} & a_{12} & a_{13} \\
    a_{21} & a_{22} & a_{23} \\
    a_{31} & a_{32} & a_{33}
  \end{pmatrix}
\end{equation}

\section{表格示例}
下面是一个简单的表格示例:

\begin{table}[htbp]
  \centering
  \caption{材料物理性质}
  \label{tab:materials}
  \begin{tabular}{lccr}
    \toprule
    材料 & 密度(kg/m$^3$) & 弹性模量(GPa) & 泊松比 \\
    \midrule
    钢铁 & 7850 & 210 & 0.3 \\
    铝 & 2700 & 70 & 0.35 \\
    铜 & 8940 & 120 & 0.33 \\
    钛 & 4500 & 110 & 0.32 \\
    \bottomrule
  \end{tabular}
\end{table}

\section{图片示例}
插入图片非常简单,下面是一个示例:

\begin{figure}[htbp]
  \centering
  % 请替换为实际的图片文件名
  \includegraphics[width=0.7\textwidth]{example-image}
  \caption{这是一个示例图片}
  \label{fig:example}
\end{figure}

多个图片排列示例:
\begin{figure}[htbp]
  \centering
  \begin{subfigure}{0.45\textwidth}
    \includegraphics[width=\textwidth]{example-image-a}
    \caption{子图A}
    \label{fig:sub-a}
  \end{subfigure}
  \hfill
  \begin{subfigure}{0.45\textwidth}
    \includegraphics[width=\textwidth]{example-image-b}
    \caption{子图B}
    \label{fig:sub-b}
  \end{subfigure}
  \caption{包含两个子图的图片示例}
  \label{fig:subfigures}
\end{figure}

\begin{figure}[htbp]
    \centering
    \includegraphics[width=0.7\textwidth]{/Users/henri/Downloads/WX20250324-161749@2x.png}
    \caption{示例图片}
    \label{fig:Hinton 讲座}
\end{figure}

\section{列表示例}
\subsection{无序列表}
以下是一个无序列表的例子:
\begin{itemize}
  \item 列表项1
  \item 列表项2
  \item 列表项3,还可以有子列表:
  \begin{itemize}
    \item 子列表项1
    \item 子列表项2
  \end{itemize}
\end{itemize}

\subsection{有序列表}
以下是一个有序列表的例子:
\begin{enumerate}
  \item 第一步
  \item 第二步
  \item 第三步,还可以有子列表:
  \begin{enumerate}
    \item 子步骤1
    \item 子步骤2
  \end{enumerate}
\end{enumerate}

\section{参考文献引用}
引用文献的方法很简单,例如引用Donald Knuth的《The \TeX book》\cite{knuth1984texbook}或者Leslie Lamport的《\LaTeX: A Document Preparation System》\cite{lamport1994latex}。

\section{结论}
通过这个示例文档,我们展示了如何使用\LaTeX{}创建支持中文的文档,并且插入数学公式、表格和图片。希望这个模板能够帮助您开始使用\LaTeX{}进行中文文档的排版工作。

\printbibliography % 打印参考文献

\appendix
\section{附录:常用\LaTeX{}命令}
这里列出了一些常用的\LaTeX{}命令,供参考:

\begin{table}[htbp]
  \centering
  \caption{\LaTeX{}常用命令}
  \label{tab:latex-commands}
  \begin{tabular}{ll}
    \toprule
    命令 & 功能 \\
    \midrule
    \verb|\section{}| & 创建一级标题 \\
    \verb|\subsection{}| & 创建二级标题 \\
    \verb|\textbf{}| & 创建粗体文本 \\
    \verb|\textit{}| & 创建斜体文本 \\
    \verb|\cite{}| & 引用参考文献 \\
    \verb|\ref{}| & 引用标签 \\
    \bottomrule
  \end{tabular}
\end{table}

\end{document}
