% 创建一个LaTeX文件,支持中文文本编译,使用XeLaTeX编译器。
% 提供插入数学公式、表格和图片等功能。

% 首先,我们需要准备好以下文件:
% 1. 一个.tex文件,包含文档内容。
% 2. 一个.bib文件,包含参考文献。
% 3. 一个.bst文件,包含参考文献的格式。
% 4. 一个.cls文件,包含文档的格式。
% 5. 一个.pdf文件,包含编译后的文档。


% 以下是.tex文件的内容:

% 首先,我们需要声明文档的类型,这里使用ctexart类,它是中文文档的类。


\documentclass[UTF8]{ctexart}
\usepackage{graphicx}
% 然后,我们需要设置文档的标题、作者、日期等信息。
\title{我的第一个LaTeX文档}
\author{作者}
\date{\today}

\begin{document}
% 接下来,我们需要使用\maketitle命令生成标题页。
\maketitle

% 接下来,我们可以开始写正文了。
\section{第一节}

这是我的第一个LaTeX文档。

% 我们可以使用\section命令来创建章节,\subsection命令来创建小节。
\subsection{第一小节}

在第一小节中,我们可以插入数学公式:

\begin{aligned}
    &w=\int_{k<\Lambda}[Dg][DA][Dw][D\Phi]exp\left\{i\int d^{4}x\sqrt{-g}\left[\frac{m_{p}^{2}}{2}R \\
    &\left.\left.-\frac{1}{4}F_{\mu\nu}^{a}F^{a\mu\nu}+i\overline{\psi}^{i}\times^{\mu}D_{\mu}\psi^{i}+\left(\overline{\psi}^{i}_{L}V_{ij}\phi\psi_{R}^{i}+h.c.\right)-\left|D_{\mu}\phi\right|^{2}-V\left(\phi\right)\right]\right\}
\end{aligned}



\begin{verbatim}
    Fréchet Derivative
    
    Fréchet derivative of $f: \mathbb{C}^{n \times n} \rightarrow \mathbb{C}^{n \times n}$ at $X \in \mathbb{C}^{n \times n}$
    
    A linear mapping $L: \mathbb{C}^{n \times n} \rightarrow \mathbb{C}^{n \times n}$ s.t. for all $E \in \mathbb{C}^{n \times n}$
    
    $$f(X + E) - f(X) - L(X, E) = o(\|E\|).$$
    
    Example For $f(X) = X^2$ we have
    
    $$f(X + E) - f(X) = XE + EX + E^2,$$
    
    so $L(X, E) = XE + EX$.
    
    Nick Higham
    
    Matrix Functions
    
    17/28
    
    Figure 4 A Beamer slide.
    \end{verbatim}

\subsection{第二小节}

在第二小节中,我们可以插入表格:

\begin{table}[htbp]
\centering
\begin{tabular}{|c|c|c|}
\hline
第一列 & 第二列 & 第三列 \\
\hline
第一行 & 第二行 & 第三行 \\
\hline
\end{tabular}
\caption{我的第一个表格}
\end{table}

\subsection{第三小节}

在第三小节中,我们可以插入图片:

\begin{figure}[htbp]
\centering
\includegraphics[width=0.5\textwidth]{example-image}
\caption{我的第一个图片}
\end{figure}

% 最后,我们需要生成参考文献。
\bibliographystyle{plain}
\bibliography{example}
\end{document}

% 至此,我们完成了一个简单的LaTeX文档。 

% 保存为.tex文件,然后使用XeLaTeX编译器编译。   
%
% 编译命令:
% xelatex <filename>.tex
%
% 编译完成后,会生成一个.pdf文件。
%
% 注意:
% 1. 编译前请先安装LaTeX编译器,如MiKTeX、TeXLive等。
% 2. 编译前请先安装ctex宏包,使用命令:
% latexmk -xelatex <filename>.tex  
%
% 3. 编译前请先安装bibtex,使用命令:
% bibtex <filename>.aux
%
% 4. 编译前请先检查是否有错误,使用命令:
% makeglossaries <filename>.tex
%
% 5. 编译前请先检查是否有警告,使用命令:
% chktex <filename>.tex
%
% 6. 编译前请先检查是否有拼写错误,使用命令:
% aspell check <filename>.tex
%
% 7. 编译前请先检查是否有重复的公式,使用命令:
% lacheck <filename>.tex

