\documentclass[UTF8]{ctexart}
\usepackage{multirow}

\title{Principles of Prompt Writing}
\author{Awesome Prompts}
\date{\today}
\begin{document}
\maketitle




\begin{table*}[h]
\vspace{-1.3in}
\fontsize{6.8pt}{8.5}\selectfont
\setlength{\tabcolsep}{5pt} 

\hspace{-0.5in}
\begin{tabular}{|c|p{10.77cm}|c|}
\hline
\bf 类别 &
 \hspace{2in} \bf 原则 &
  \textbf{\#原则序号} \\ \hline
\multirow{13}{*}{\begin{tabular}[c]{@{}c@{}}提示结构 \\ 与清晰度\end{tabular}} &

  将预期的受众整合到提示中。 &
  2 \\  

 &
  \begin{tabular}[c]{@{}l@{}}
  \\ 使用肯定的指令,如“做”,同时避免使用否定语言,如“不要”。 \end{tabular}& 
  4 \\ 
 &
  \begin{tabular}[c]{@{}l@{}}
  \\
  使用引导词,如写“逐步思考”。   \end{tabular}&
  12 \\ 
 &
  \begin{tabular}{@{}p{10.7cm}@{}}
  \\ 使用输出引导词,即以期望输出的开头结束你的提示。 \\ 通过以预期回应的开始结束你的提示。           \end{tabular}&
  20 \\ 
 &
  \begin{tabular}[c]{@{}l@{}}
  \\
  使用分隔符。  \end{tabular}&
  17 \\
 &
   \begin{tabular}{@{}p{10.7cm}@{}} \\ 当格式化你的提示时,以“\#\#\#指令\#\#\#”开始,随后是“\#\#\#示例\#\#\#”或“\#\#\#问题\#\#\#”,如果相关的话。随后,呈现你的内容。使用一个或多个换行符来分隔指令、示例、问题、背景和输入数据。\end{tabular} &
  8 \\ \hline

  
\multirow{26}{*}{\begin{tabular}[c]{@{}c@{}}具体性 \\ 和信息\end{tabular}} &

\begin{tabular}[c]{@{}l@{}}
  实施示例驱动的提示(使用少量示例提示)。 \end{tabular} &
  7 \\
 &
 \begin{tabular}{@{}p{10.7cm}@{}}
\\ 当你需要清晰理解一个主题、想法或任何信息时,使用以下提示:\\ \hspace{0.3cm} o 用简单的话解释[插入特定主题]。 \\  \hspace{0.3cm} o 像我是11岁一样向我解释。 \\ \hspace{0.3cm} o 像我是[领域]的初学者一样向我解释。 \\ \hspace{0.3cm} o “用简单的英语写[文章/文本/段落],就像你在向一个5岁的孩子解释东西一样。”\end{tabular}       &
  5  \\
  
 &  \begin{tabular}[c]{@{}l@{}}
  \\ 在你的提示中加入以下短语“确保你的回答是无偏见的,避免依赖刻板印象。”  \end{tabular}   &
  13 \\
  
 & \begin{tabular}[c]{@{}l@{}}
  \\ 要写任何与提供的样本相似的文本,包括具体指令:\\ \hspace{0.3cm} o “根据提供的段落[/标题/文本/文章/回答]使用相同的语言。” \end{tabular}  &
  26 \\
 & \begin{tabular}{@{}p{10.7cm}@{}}
  \\ 当你想使用特定的单词、短语或句子开始或继续一个文本时,使用提供的提示结构:\\ \hspace{0.3cm} o 我为你提供了开头[歌词/故事/段落/文章...]:[插入歌词/单词/句子]。 \\ \hspace{0.45cm}  根据提供的词汇完成它。保持流畅一致。 \end{tabular} &
  24 \\
 &  \begin{tabular}{@{}p{10.7cm}@{}}
  \\
  明确说明模型必须遵循的要求,以便生成内容,以关键词、规则、提示或指令的形式。 \end{tabular}&
  25 \\
 &   \begin{tabular}{@{}p{10.7cm}@{}}
  \\
  询问特定主题或想法并测试你的理解,你可以使用以下短语[16]:\\ \hspace{0.3cm} o “教我[任何定理/主题/规则名称]并在最后包括一个测试,让我知道我回答后的答案是否正确,不要提前提供答案。” \end{tabular} &
  15 \\
 &
  \begin{tabular}[c]{@{}l@{}}\\为了写一篇详细的文章/文本/段落/文章或任何类型的文本: \\\hspace{0.3cm} o “为我详细写一篇关于[主题]的详细[文章/文本/段落],添加所有必要的信息。”\end{tabular} &
  21 \\ \hline
\multirow{4}{*}{\begin{tabular}[c]{@{}c@{}}用户交互 \\ 和参与\end{tabular}} &

 \begin{tabular}{@{}p{10.7cm}@{}}
允许模型通过向你提问来获取精确的细节和要求,直到它有足够的信息提供所需的输出
\\\hspace{0.3cm} o “从现在开始,我希望你向我提问以...'' \end{tabular} &
  14 \\
 &
   \begin{tabular}{@{}p{10.7cm}@{}} \\为了写一篇详细的文章/文本/段落/文章或任何类型的文本:“为我详细写一篇关于[主题]的详细[文章/文本/段落],添加所有必要的信息。”\end{tabular} &
  21 \\ \hline
  
\multirow{16}{*}{\begin{tabular}[c]{@{}c@{}}内容和 \\ 语言风格\end{tabular}} &
   \begin{tabular}{@{}p{10.7cm}@{}}  纠正/更改特定文本而不改变其风格:“尝试修改用户发送的每个段落。你应该只改进用户的语法和词汇,并确保听起来自然。你应该保持原始写作风格,确保正式段落保持正式。”\end{tabular}  &
  22 \\
 &
 \begin{tabular}[c]{@{}l@{}} \\
  加入以下短语:“你的任务是”和“你必须”。  \end{tabular}&
  9 \\
 & 
 \begin{tabular}[c]{@{}l@{}} \\
  加入以下短语:“你将受到惩罚。”\end{tabular}  &
  10\\
 & \begin{tabular}[c]{@{}l@{}} \\
  给语言模型分配一个角色。 \end{tabular}  &
  16 \\
 &\begin{tabular}[c]{@{}l@{}} \\
  在你的提示中使用短语“以自然语言形式回答问题”。 \end{tabular} &
  11 \\
 & \begin{tabular}{@{}p{10.7cm}@{}} \\
  与LLM交流时不需要礼貌,所以不需要添加“请”、“如果你不介意”、“谢谢”、“我想要”等短语,直接切入正题。  \end{tabular} &
  1 \\
 &\begin{tabular}[c]{@{}l@{}} \\
  在一个提示中多次重复特定的单词或短语。  \end{tabular}&
  18  \\  
 &\begin{tabular}[c]{@{}l@{}} \\
  加入“我会为更好的解决方案支付\$xxx的小费!”\end{tabular}&
  6 \\ \hline
\multirow{8}{*}{\begin{tabular}[c]{@{}c@{}}复杂任务和 \\ 编码提示\end{tabular}} 
&
  将复杂任务分解成一系列简单的提示,在交互式对话中进行。  &
  3 \\
 &
  \begin{tabular}{@{}p{10.7cm}@{}} \\
  当你有一个可能涉及不同文件的复杂编码提示时:\\\hspace{0.3cm} o “从现在开始,每当你生成跨越多个文件的代码时,生成一个可以运行的[编程语言]脚本,以自动创建指定的文件或对现有文件进行更改以插入生成的代码。[你的问题]。”\end{tabular}&
  23 \\
 & \begin{tabular}[c]{@{}l@{}} \\
  结合思维链(Cot)和少量示例提示。\end{tabular} & 
  19 \\ \hline
\end{tabular}
\caption{提示原则类别。 }
\
\label{tab:categories}
\end{table*}

\end{document}
